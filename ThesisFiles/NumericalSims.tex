\chapter{Numerical simulations of nonlinear optics}
 \label{app:numericalsims}



This appendix describes the algorithm used for numerical simulation of the nonlinear Schrodinger equation (NLSE) and Lugiato-Lefever equation (LLE) to obtain the results presented in the preceding chapters in this thesis. These equations are simulated with Matlab using a fourth-order Runge-Kutta interaction picture (RK4IP) method \cite{Hult2007} with adaptive step size \cite{Heidt2009}.

\section{RK4IP algorithm}

The LLE (NLSE) describes the evolution of the field $\psi$ ($A$), a function of a fast variable $\theta$ ($T$), over a timescale parametrized by a slow variable $\tau$ ($z$). In what immediately follows we use the variable names corresponding to the LLE for simplicity. Each of these equations can be written as the sum of a nonlinear term $\hat{N}$ and a linear operator $\hat{L}$, so that the field $\psi$ evolves as:
\begin{equation}
\frac{\partial\psi}{\partial\tau}=(\hat{N}+\hat{L})\psi.
\end{equation}

The RK4IP algorithm specifies a recipe for advancing the field a single step $h$ in the slow variable $\tau$ to obtain $\psi(\theta,\tau+\delta)$ from $\psi(\theta,\tau)$. The recipe is:
\begin{align}
\psi_I=&\exp\left(\frac{\delta}{2}\hat{L}\right)\psi(\theta,\tau) \\
k_1=&\exp\left(\frac{\delta}{2}\hat{L}\right)\left[\delta\tau\hat{N}(\psi(\theta,\tau))\right]\psi(\theta,\tau) \\
k_2=&\delta\hat{N}(\psi_I+k_1/2)\left[\psi_I+k_1/2\right] \\
k_3=&\delta\hat{N}(\psi_I+k_2/2)\left[\psi_I+k_2/2\right] \\
k_4=&\delta\hat{N}\left(\exp\left(\frac{\delta}{2}\hat{L}\right)(\psi_I+k_3)\right) \\
&\times\exp\left(\frac{\delta}{2}\hat{L}\right)(\psi_I+k_3)\\
\psi(\theta,\tau+\delta)=&\exp\left(\frac{\delta}{2}\hat{L}\right)[\psi_I+k_1/6+k_2/3+k_3/3]+k_4/6.
\end{align}

Simulation according to the RK4IP method is carried out using a split-step Fourier approach, where $\hat{L}$ is applied in the frequency domain and $\hat{N}$ is applied in the time domain.

\section{Adaptive step-size algorithm}

An adaptive step-size algorithm is a strategy for adjusting the magnitude of the steps $\delta$ that are taken to optimize the simulation speed while maintaining a desired degree of accuracy. The RK4IP algorithm exhibits error that scales locally as $O(\delta^5)$. Since reducing the step size naturally requires more steps, increasing the number of small errors that accumulate, the resulting global accuracy of the algorithm is $O(\delta^4)$. The appropriate step-size adjustment algorithm for this scaling is described by Heidt \cite{Heidt2009}. For a given desired maximum error $e_G$ (goal error) from step to step, the algorithm goes as follows:
\begin{itemize}
	\item Calculate a field $\psi_{coarse}$ by advancing the field $\psi(\theta,\tau)$ according to RK4IP by a step of size $\delta$.
	\item Calculate a field $\psi_{fine}$ by advancing the field $\psi(\theta,\tau)$ according to RK4IP by two steps of size $\delta/2$. 
	\item Calculate the measured error $e=\sqrt{\sum_j |\psi_{coarse,j}-\psi_{fine,j}|^2/\sum_j|\psi_{fine,j}|^2}$, where $j$ indexes over the discrete points parametrizing the fast variable $\theta$. 
	\begin{itemize}
		\item If $e>2e_G$, discard the solution and repeat the process with coarse step size $\delta'=\delta/2$.
		\item If $e\in(e_G,2e_G)$, the evolution continues and the step size is reduced to $\delta'=\delta/2^{1/5}\approx0.87\delta$. 
		\item If $e\in(e_G/2,e_G)$, the evolution continues and the step size is not changed.
		\item If $e<e_G/2$, the evolution continues and the step size is increased to $\delta'=2^{1/5}\delta\approx1.15\delta$.
		\end{itemize}
\end{itemize}

When the simulation continues, the new field $\psi(\theta,\tau+\delta)$ is taken to be $\psi(\theta,\tau+\delta)=16\psi_{fine}/15-\psi_{coarse}/15$. In the calculations described in this thesis, the goal error $e_G$ is typically $10^{-6}$.

\section{Pseudo-code for numerical simulation with the RK4IP algorithm and adaptive step size}


The following block of pseudo-code shows how the RK4IP algorithm with adaptive step-size is implemented. The pseudo-code neglects the details of the RK4IP algorithm, in which the operator $\hat{L}$ is applied in the frequency domain and the operator $\hat{N}$ is applied in the time domain. 

Several things to note:
\begin{itemize}
	\item $\psi$ is not overwritten until the new solution is acceptable. 
	\item This implementation makes use of an extra efficiency that is possible when the solution is discarded and the step-size is halved: the first step of the fine solution $\psi_{fine,1}$ for the previous attempt becomes the coarse solution $\psi_{coarse}$ for the current step.
\end{itemize}


	\begin{algorithmic}
		

		
		
		
		\Procedure{Run the loop}{}
		\While{$\tau<\tau_{end}$}
		
		\State $e=1$ initialize the error to a large value
		\State $firsttry=TRUE$ for more efficiency on the first step, see below
		\State $\delta=2\delta$ to account for halving on the first iteration
		
		\While{$e>2e_G$}
		\If{$firsttry$} \State $\psi_{coarse}=\mathrm{RK4IP}(\psi,\delta)$
		\Else \State $\,\psi_{coarse}=\psi_{fine,1}$
		\EndIf
		\State$\delta=\delta/2$
		\State$\psi_{fine}=\psi$
		\For{$j_{step}=1:2$}
		\State $\psi_{fine}=\mathrm{RK4IP}(\psi_{fine},\delta)$
			\If{$j_{step}=1$}
		\State $\psi_{fine,1}=\psi_{fine}$
		\EndIf
		\EndFor
		\State $e=\sqrt{\sum{|\psi_{coarse}-\psi_{fine}|^2}/\sum{|\psi_{fine}|^2}}$
		\State $firsttry=FALSE$
		\EndWhile
		
		\EndWhile\\
		\State $\psi=16\psi_{fine}/15-\psi_{coarse}/15$
		\If{$e>e_G$} \State{$\delta=\delta/2^{1/5}$}
		\EndIf
		\If{$e<e_G/2$} \State{$\delta=2^{1/5}\delta$}
		\EndIf
	\EndProcedure
	\end{algorithmic}

\subsection{Simulation of the LLE}
For simulation of the LLE, the operators are: 
\begin{align}
\hat{N}&=i|\psi|^2+F/\psi,\\
\hat{L}&=-(1+i\alpha_\mu),\\
\alpha_\mu&=\alpha-\beta_2\mu^2/2,
\end{align}
where the subscript $\mu$ indicates the pump-referenced mode number upon which the operator acts. Note, in particular, that the pump term $F$ has been incorporated into the nonlinear operator, so that it is implemented in the time domain. The quantity $\hat{N}\psi$ then becomes $i|\psi|^2\psi+F$, as required for computation of $\partial\psi/\partial\tau$.

\subsection{Simulation of the NLSE}





%		\begin{align}
%		\psi_I=&\exp\left(\frac{\delta}{2}\hat{L}\right)\psi\\
%		k_1=&\exp\left(\frac{\delta}{2}\hat{L}\right)\left[\delta\tau\hat{N}(\psi)\right]\psi \\
%		k_2=&\delta\hat{N}(\psi_I+k_1/2)\left[\psi_I+k_1/2\right] \\
%		k_3=&\delta\hat{N}(\psi_I+k_2/2)\left[\psi_I+k_2/2\right] \\
%		k_4=&\delta\hat{N}\left(\exp\left(\frac{\delta}{2}\hat{L}\right)(\psi_I+k_3)\right) \\
%		&\times\exp\left(\frac{\delta}{2}\hat{L}\right)(\psi_I+k_3)\\
%		\psi_{coarse}=&\exp\left(\frac{\delta}{2}\hat{L}\right)[\psi_I+k_1/6+k_2/3+k_3/3]+k_4/6.
%		\end{align}


	
%		\begin{align}
%		\psi_I=&\exp\left(\frac{\delta}{2}\hat{L}\right)\psi_{step}\\
%		k_1=&\exp\left(\frac{\delta}{2}\hat{L}\right)\left[\delta\tau\hat{N}(\psi_{step})\right]\psi \\
%		k_2=&\delta\hat{N}(\psi_I+k_1/2)\left[\psi_I+k_1/2\right] \\
%		k_3=&\delta\hat{N}(\psi_I+k_2/2)\left[\psi_I+k_2/2\right] \\
%		k_4=&\delta\hat{N}\left(\exp\left(\frac{\delta}{2}\hat{L}\right)(\psi_I+k_3)\right) \\
%		&\times\exp\left(\frac{\delta}{2}\hat{L}\right)(\psi_I+k_3)\\
%		\psi_{fine}=&\exp\left(\frac{\delta}{2}\hat{L}\right)[\psi_I+k_1/6+k_2/3+k_3/3]+k_4/6.\\
%		\end{align}



%		\Procedure{Initialize}{}\\
%		$e=10^{-6}$\\
%		$\delta=10^{-3}$\\
%		\\
%		$F^2=5$, $F=\sqrt{F^2}$\\
%		$\alpha=0.95\times\pi^2F^2/8$\\
%		$\beta_2=-0.02$\\
%		\\
%		$\tau=0$\\
%		$\tau_{end}=2000$\\
%		\\
%		$N_{pnts}=2^{12}$\\
%		$d\theta=2\pi/N_{pnts}$\\
%		$\theta=[-\pi:d\theta:\pi-d\theta]$
%		\\
%		$k=[0, 1, ...,N_{pnts}/2-1, -N_{pnts}/2, -N_{pnts}/2+1, ..., -1]$\\
%		$\alpha_k=\alpha-\beta_2 k^2/2$\\
%		\\
%		$\hat{L}=-(1+i\alpha_k)$\\
%		$\hat{N}(\psi)\psi\equiv i|\psi|^2\psi+F$, defining a function for the nonlinear operator and incorporating the pump\\
%		\\
%		$\psi=\psi_{s,min}+\sqrt{2\alpha}e^{i\phi_0}\mathrm{sech}\left(\sqrt{\frac{2\alpha}{-\beta_2}}\theta\right)$, to initialize at an approximate soliton solution
