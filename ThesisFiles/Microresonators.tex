\chapter{Introduction to microresonator-based frequency combs}
 \label{ch:microresonators}
 
 This chapter introduces the basic physics of  optical frequency-comb generation in Kerr-nonlinear microring resonators and discusses some preliminary considerations for the applications of these combs, and the two \todo{correct?} subsequent chapters describe results obtained in these systems.
 
 \section{Optical microring resonators}
An optical microring resonator, shown schematically in Fig. \ref{fig:RingResonator}, guides light around a closed path in a dielectric medium by total internal reflection. The principle is the same as the guiding of light in an optical fiber, and indeed a `macroring' resonator can be constructed from a loop of fiber. These resonators are sometimes referred to as whispering-gallery mode resonators due to the similarity between their guided modes and the acoustic `whispering-gallery' waves that permit an observer on one side of St. Paul's cathedral to hear whispers uttered by a speaker on the other side of the cathedral, as explained by Lord Rayleigh beginning in 1910. Optical microring resonators have a host of characteristics that make them useful for nonlinear optics and photonics applications; these include the ease with which they can be integrated, the ultra-high Q factors that have been demonstrated, and the ability to tailor the spectral distribution of guided modes through careful resonator design.  


A microring resonator supports propagating guided optical modes of electromagnetic radiation with (vacuum) wavelengths that evenly divide the optical round-trip path length: $\lambda_m=n_{eff}(\lambda_m)L/m$, with associated resonance frequencies $\nu_m=c/\lambda_m=mc/n_{eff}(\nu_m)L$. This leads to constructive interference from round-trip to round-trip. Here $L$ is the physical circumference of the resonator, $m$ is the azimuthal mode number, and $n_{eff}(\lambda_m)$ is an effective index of refraction that depends on the resonator geometry and the transverse intensity profile of the mode (see e.g. \cite{REFHERE} for more information). The free-spectral range $f_{FSR}$ of a resonator is the \textit{local} frequency spacing between modes, calculated via:
\begin{align}
	f_{FSR}&\approx \nu_{m+1}-\nu_{m}\approx \nu_{m}-\nu_{m-1},\\
	&=\frac{\partial\nu_m}{\partial m},\\
	&=\frac{c}{n_{eff}(\nu)L}-\frac{mc}{n_{eff}^2(\nu)L}\frac{\partial n_{eff}}{\partial \nu}\frac{\partial \nu}{\partial m},\\
	\Rightarrow f_{FSR}&=\frac{c/L}{\left(n_{eff}+\nu\frac{\partial n_{eff}}{\partial \nu}\right)}=\frac{c}{n_g L}=1/T_{RT},
\end{align}
	where $n_g=n_{eff}+\nu\frac{\partial n_{eff}}{\partial \nu}$ is the group velocity of the mode and $T_{RT}$ is the mode's round-trip time. Importantly, both intrinsic material dispersion and geometric dispersion resulting from, e.g., different sampling of core versus cladding material properties for different transverse mode profiles, lead to a frequency dependence for each of the parameters $n_{eff}$, $n_g$, and $f_{FSR}$, and a resulting non-uniform spacing in the cavity modes in frequency despite the linearity of $\nu_m$ in $m$. 
	
	 Depending on the design, microring resonators can support a single propagating transverse mode profile or may be multi-mode, meaning that many different transverse mode profiles are supported. The former can be readily achieved using chip-integrated photonic waveguides that provide index contrast and transverse confinement on four sides; the latter is typical of resonators that lack an inner radius dimension and therefore exhibit less spatial confinement, such as free-standing silica microrod resonators \cite{DelHaye2013}. For a given resonator geometry, to calculate the frequency-dependent effective index $n_{eff}(\nu)$, thereby enabling calculation of the resonance frequencies and wavelengths, one must solve Maxwell's equations. Except in special cases of high symmetry \cite{Oraevsky2002}, this is typically done numerically using finite-element modeling tools like COMSOL. The modes of an optical resonator, both within a mode family defined by a transverse mode profile and between mode families, must be orthogonal\cite{Haus1984}.

\subsection{Resonant enhancement in a microring resonator}
 The lifetime $\tau_\gamma$ of circulating photons in a resonator is fundamental to its fitness for applications. Generally, two processes lead to the loss of circulating photons: intrinsic dissipation that occurs at a rate $1/\tau_{int}$ and outcoupling to an external waveguide that occurs at a rate $1/\tau_{ext}$, leading to a total loss rate of $\tau_\gamma^{-1}=\tau_{ext}^{-1}+\tau_{int}^{-1}$. To understand the quantitative role of these parameters, we consider a cavity mode of frequency $\omega_0$ and amplitude $a$ (normalized such that $|a|^2=N$, the number of circulating photons) driven by a field with frequency $\Omega_0$ and rotating amplitude $s\propto\exp(i\omega t)$ (normalized such that $|s|^2=S$, the rate at which photons in the coupling waveguide pass the coupling port) that is in-coupled with strength $\kappa$. The equation of motion for such a system is\cite{Haus1984}:
 \begin{equation}
 \frac{d a}{d t}=i\omega_0 a-\left(\frac{1}{2\tau_{int}}+\frac{1}{2\tau_{ext}}\right)a+\kappa s. \label{eq:coupledmotion}
 \end{equation}
 We can immediately solve this equation by assuming that $a\propto\exp(i\Omega_0 t)$, and we obtain:
 \begin{equation}
 a=\frac{\kappa s}{\left(\frac{1}{2\tau_{int}}+\frac{1}{2\tau_{ext}}\right)+i(\Omega_0-\omega_0)}. \label{eq:coupledsoln}
 \end{equation}
 
 To extract anything further from this equation, we must derive a relationship between $\tau_{ext}$ and $\kappa$, which so far are not related. To do this, we exploit the time-reversal symmetry that is inherent in this system when there is no dissipation, that is, when $1/\tau_{int}=0$. In the case of only an initial excitation $N_0$ decaying into the waveguide with the driving term $s$ set to zero, we have $N=N_0e^{-t/\tau_{ext}}$. In this case, energy conservation guarantees that the rate $S_{out}$ at which photons propagate away from the resonator in the waveguide is $-dN/dt=N_0e^{-t/\tau_{ext}}/\tau_{ext}$; we therefore have $S_{out}=N/\tau_{ext}$. In the time-reversed system with $t\rightarrow-t$, this amplitude $S_{out}$ describes the rate of pumping: the cavity is resonantly driven with increasing power $S=S_{out}(-t)=N_0^{t/\tau_{ext}}/\tau_{ext}$. In this case the frequency of the driving field $s$ can be written $\omega_0-i/2\tau_{ext}$. Inserting this frequency into Eq. \ref{eq:coupledsoln} gives the equations
 \begin{equation}
  a=\kappa s \tau_{ext}
  \end{equation}
  and
  \begin{equation}
   N=|\kappa|^2 S \tau_{ext}^2.
   \end{equation}
   By comparing the relationships between $S_{out}$ and $N$ for the forward-evolving system and $S$ and $N$ for the backward-evolving system, we arrive at the relationship $|\kappa|^2=1/\tau_{ext}$. This relationship holds generally, and we can return to the case including dissipation and insert it into Eq. \ref{eq:coupledsoln}, which can then be squared to obtain:
   \begin{equation}
   N=\frac{\Delta\omega_{ext}S}{\Delta\omega_{tot}^2/4+(\Omega_0-\omega_0)^2} \label{eq:resenhancement}
   \end{equation}
   Here we define the linewidths $\Delta\omega_{ext}=1/\tau_{ext}$, $\Delta\omega_{int}=1/\tau_{int}$, and $\Delta\omega_{tot}=\Delta\omega_{ext}+\Delta\omega_{int}$. Two important observations can be drawn from Eq. \ref{eq:resenhancement}: First, the cavity response is Lorentzian with a full-width at half-maximum (FWHM) linewidth that is related to the photon lifetime via $\tau_\gamma=1/\Delta\omega_{tot}$, and second, on resonance the number of circulating photons is related to the input rate by the factor $\Delta\omega_{ext}/\Delta\omega_{tot}^2$. The combination of this resonant enhancement and a small cavity mode volume enables very large circulating optical intensities, which is important for the application of microresonators in nonlinear optics.
   
   Two commonly used practical quantities are linked to the photon lifetime: the resonator finesse $\mathcal{F}=2\pi\tau_\gamma/T_{RT}$, which for a ring resonator can be interpreted literally as the azimuthal resonator angle traced out by a typical photon over its lifetime; and the resonator quality factor $Q=\omega_c \tau_\gamma$, the phase over which the optical field evolves during the photon lifetime. Using the relationship $\tau_\gamma=1/\Delta\omega_{tot}$, the finesse and quality factor can be rewritten as simple ratios of the relevant frequencies: $\mathcal{F}=f_{FSR}/\Delta\nu$; $Q=\nu_c/\Delta\nu$, where $\Delta\nu=\Delta\omega_{tot}/2\pi$.
   
   
  
 
 
% 
% 
% Energy conservation ensures that the rate at which photons propagate away from the resonator in the coupling waveguide is $|s|^2=S=-dN/d t=\frac{N_0}{\tau_{ext}}\exp(-t/\tau_{ext})$. We can consider the time-reversed system in which the input increases in magnitude as $S=\frac{N_0}{\tau_{ext}}\exp(t/\tau_{ext})$ and is resonant with the cavity mode. In this case, the frequency of the drive $s$ is $\omega_s=\omega_0-i/2\tau_{ext}$. Inserting this frequency into Eq. \ref{eq:coupledsoln} for the case of $\Delta\omega_{int}=0$ yields:
% However, for the system without time reversal we have $N=N_0\exp(-t/\tau_{ext})$, $S=\frac{N_0}{\tau_{ext}}\exp(-t/\tau_{ext})$, so that $N=\tau_{ext} S$. By inserting this into the previous equation, we obtain . 
% This is quantified by the basic relation for the number of circulating photons $N(t)=N_oe^{-t/\tau_\gamma}$ in the presence of solely linear loss, which defines the photon lifetime $\tau_\gamma$. 
%
%\begin{equation}
%\mathcal{F}\{E\}(\omega)\propto\int_0^\infty dt\, e^{-\left(\frac{1}{2\tau_\gamma}+i(\omega_c-\omega)\right)t},
%\end{equation}
%which immediately yields the Lorentzian lineshape
%\begin{equation}
%|\mathcal{F}\{E\}|^2\propto\frac{1}{(\omega-\omega_c)^2+\frac{1}{4\tau_\gamma^2}}, \label{eq:lorentzian}
%\end{equation}
%with FWHM linewidth $\Delta\omega=1/\tau_\gamma$. With this relationship, 
%
%The utility of a resonator with long photon lifetime is illustrated by calculating the steady-state number of circulating photons $N$ in a system including a driving term $S$ denoting the rate at which photons are added\cite{Haus1984}. Working in the field quantities $a$ ($|a|^2=N$) and $s$ ($|s|^2=S$), the equation of motion for such a system is: 
%
%The terms in this equation represent the rotation of the field at the mode's frequency $\omega_0$, the strength $\kappa$ of coupling from the input waveguide to the resonator, and the intrinsic and extrinsic (power) loss timescales $\tau_{int}=1/\Delta\omega_{int}$ and $\tau_{ext}=1/\Delta\omega_{ext}$ due to loss and outcoupling. If the drive has frequency $\omega_s$, 
%
%The rates $\Delta\omega_{ext}$ and $\kappa$ are not \textit{a priori} related, but we can derive the relationship between them by exploiting the time-reversal symmetry of systems without dissipation. For the case of no pumping ($s=0$) and no losses ($\Delta\omega_{int}=0$), we have 
%\begin{align}
%\frac{d a}{dt}&=i\omega_0 a-\frac{a}{2\tau_{ext}},\\
%\Rightarrow &a=a_0\exp(i\omega_0 t-t/2\tau_{ext}), \\
%&N=N_0\exp(-t/\tau_{ext}).
%\end{align}





%For the case without dissipation, then, we know that $N=S\tau_ext$, which is the solution of the rate equation
%\begin{equation}
%\frac{dN}{dt}=-N
%\end{equation}
%
%We can now derive a rate equation describing $N(t)$ from Eq. \ref{eq:coupledmotion}:
%\begin{align}
%\frac{dN}{dt}=\frac{d|a|^2}{dt}=a^*\frac{da}{dt}+a\frac{da^*}{dt}
%\end{align}





%Thus the steady-state number of circulating photons and therefore the intensity is larger than the input rate by a factor of the photon lifetime. Of course a more complete calculation including effects such as outcoupling at the coupler (with a rate related to the driving term $A$) could be performed, but this simple calculation captures the essence of the importance of photon lifetime $\tau_\gamma$. \todo{I am not sure whether this captures the scaling that I want to indicate}

\subsection{Thermal effects in WGM resonators} \label{sec:thermaleffects}

%In a typical microresonator frequency-comb experiment, a frequency-tunable pump laser is coupled evanescently into and out of the resonator using a tapered optical fiber (for e.g. free-standing silica disc resonators) or a bus waveguide (for chip-integrated resonators, e.g. in silicon nitride rings). This thesis describes experiments in which the wavelength of the pump laser is always in the telecommunications band, near $\lambda=$1550 nm. However, other pump wavelengths are possible, and frequency combs have been generated with pumps ranging from the visible \cite{visiblecombs} to the outer reaches of the near infrared \cite{midIRcombs}\todo{true?}. When overlap between the evanescent mode of the fiber and a whispering-gallery mode of the resonator is achieved, with the frequency of the pump laser close to the resonant frequency of that mode, light will build up in the resonator and the transmission of the pump laser past the resonator will decrease.

In a typical microresonator frequency-comb experiment, a frequency-tunable pump laser is coupled evanescently into and out of the resonator using a tapered optical fiber (for e.g. free-standing silica disc resonators) or a bus waveguide (for chip-integrated resonators, e.g. in silicon nitride rings). When overlap between the evanescent mode of the fiber and a whispering-gallery mode of the resonator is achieved, with the frequency of the pump laser close to the resonant frequency of that mode, light will build up in the resonator and the transmission of the pump laser past the resonator will decrease.

In any experiment in which a significant amount of pump light is coupled into a resonator, one immediately observes that the cavity resonance lineshape in a scan of the pump-laser frequency is not Lorentzian as expected from Eq. \ref{eq:resenhancement}. This is due to heating of the resonator as it absorbs circulating optical power. Since the mode-field volume and the physical volume of the microresonator are both small, thermal effects are large enough to have important practical implications in microresonator experiments. As the local volume of the mode heats (over a `fast thermal timescale') and this energy is conducted to and heats the rest of the resonator (over the `slow thermal timescale') \cite{Ilchenko1992}, the resonance frequency of a given cavity mode shifts due to the thermo-optic coefficient $\partial n/\partial T$ and the coefficient of thermal expansion of the mode volume $\partial V/\partial T$. For typical microresonator materials the thermo-optic effect dominates, and $\partial n/\partial T>0$ leads to a decrease in the resonance frequency with increased circulating power in steady state.

A calculation of the thermal dynamics of the system composed of the pump laser and the resonator reveals that there is a range of pump-laser frequencies $\Omega_0$ (which depends on the pump laser power) near and below the `cold-cavity' resonance frequency of a given cavity mode over which the system has three possible thermally-shifted resonance frequencies $\omega_{0,shifted}$ at which thermal steady state is achieved. Generally, these points are\cite{Carmon2004}:
\begin{enumerate}
\item $\Omega_o>\omega_{0,shifted}$,
\item $\Omega_o<\omega_{0,shifted}$,
\item $\Omega_o\ll\omega_{0,shifted}$.
\end{enumerate}
These points correspond to the case of (1) Blue detuning with significant coupled power and a thermal shift of the resonance, (2) Red detuning with significant coupled power and a thermal shift of the resonance, and (3) Large red detuning with no significant coupled power and no thermal shift of the resonance. Steady-state point (1) is experimentally important, because in the presence of pump-laser frequency and power fluctuations it leads to so-called thermal `self-locking.' Specifically for steady-state point (1), this can be seen as follows: 
\begin{itemize}
	\item If the pump-laser power increases, the cavity heats, the resonance frequency decreases, the detuning increases, and the change in coupled power is minimized.
	\item If the pump-laser power decreases the cavity cools, the resonance frequency increases, the detuning decreases, and the change in coupled power is minimized.
	\item If the pump-laser frequency increases the cavity cools, the resonance frequency increases, and the change in detuning is minimized.
	\item If the pump-laesr frequency decreases the cavity heats, the resonance frequency decreases, and the change in detuning is minimized.
	\end{itemize}
This is in contrast with steady-state point (2), where each of the four pump-laser fluctuations considered above generates a positive feedback loop, with the result that any fluctuation will lead the system to flip to steady-state point (1) or (3). This preference of the system to occupy point (1) or point (3) over a range of red detuning is referred to as thermal bi-stability. One consequence is that the transmission profile of the pump laser takes on hysteretic behavior in a scan over a cavity resonance with significant pump power: in a decreasing frequency scan, the lineshape takes on a broad sawtooth shape, while in an increasing frequency scan, the resonance takes on a narrow pseudo-Lorentzian profile whose exact shape depends on the scan parameters relative to the thermal timescale. This is shown in Fig. \ref{fig:thermaltriangle}. \color{red}A second consequence is that operation at red detuning with significant coupled power in a microresonator experiment requires special efforts to mitigate the effects of thermal instability.\color{black}








\section{Microring resonator Kerr frequency combs}

The high circulating optical intensities accessible in resonators with long photon lifetimes find immediate application in the use of microresonators for nonlinear optics. The experiments described in this thesis are conducted in silica microresonators. Silica falls into a broader class of materials that exhibit both centro-symmetry, which dictates that the second-order nonlinear susceptibility $\chi^{(2)}$ must vanish, and a significant third-order susceptibility $\chi^{(3)}$. The $n$\textsuperscript{th}-order susceptibility is a term in the Taylor expansion describing the response of the medium's polarization to an external electric field\cite{Boyd2003}: $P=P_0+\epsilon_0 \chi^{(1)} E + \epsilon_0 \chi^{(2)} E^2 + \epsilon_0 \chi^{(3)} E^3+...$. The effect of $\chi^{(3)}$ can be described in a straightforward way as a dependence of the refractive index on the local intensity\cite{Agrawal2007},
\begin{equation}
n=n_0+n_2 I \label{eq:KerrIndex}
\end{equation}
where $n_2=\frac{3\chi^{(3)}}{4n_0^2\epsilon_0 c}$\cite{DelCoso2004,Agrawal2007}. The intensity-dependence of the refractive index resulting from the third-order susceptibility $\chi^{(3)}$ is referred to as the optical Kerr effect.

The combination of the Kerr effect and the high circulating intensities that are accessible in high-finesse cavities provides a powerful platform for nonlinear optics. Specifically, the Kerr effect enables self-phase modulation, cross-phase modulation, and four-wave mixing, the last of which is depicted schematically in Fig. \ref{fig:FWM}. 

In 2007, a remarkable result heralded the beginning of a new era for frequency comb research. Del'Haye et al reported \textit{cascaded four-wave mixing} in toroidal silica microcavities on silicon chips, the result of which was a series of many co-circulating optical fields that were uniformly spaced by $f_{rep}$ ranging from 375 GHz to $\sim$750 GHz (depending on the platform)\cite{DelHaye2007}. Measurements indicated that the frequency spacing was uniform to a precision of $7.3 \times 10^{-18}$, thereby establishing that the output of the system was a frequency comb. This result built on previous demonstrations of few-mode parametric oscillations in microresonators \cite{Kippenberg2004, Savchenkov2004,Agha2007}, and showed that the non-uniform distribution of cavity resonance frequencies due to dispersion could be overcome to generate an output with equidistant frequency modes. Demonstrations of frequency-comb generation in other platforms followed shortly, with realizations in ... A second important development occurred in 2012, when Herr et al reported the generation of frequency combs corresponding in the time domain to single circulating optical `soliton' pulses with duration less than the cavity round-trip time. This observation followed the observation of solitons in formally-equivalent passive fiber-ring resonators in 2010\cite{Leo2010a}. Due to their unique properties, as discussed in Sec. \ref{sec:LLEsolitons}, the generation and manipulation of soliton combs has become a significant priority in microcomb research. 

\section{A model for Kerr-comb nonlinear optics: The Lugiato-Lefever equation}

Kerr-comb generation can be motivated and partially understood through the CFWM picture \cite{Herr2012}, but the phase and amplitude degrees of freedom for each comb line mean that CFWM gives rise to a rich space of comb phenomena---it is now known that Kerr combs can exhibit several fundamentally distinct outputs.  A useful model for understanding this rich space is the Lugiato-Lefever equation (LLE), which was shown to describe microcomb dynamics by Chembo and Menyuk \cite{Chembo2013} through Fourier-transformation of a set of coupled-mode equations describing CFWM and by Coen et al \cite{Coen2013a} through time-averaging of an Ikeda map for a low-loss resonator (as first performed by Haelterman, Trillo, and Wabnitz \cite{Haelterman1992a}).  The LLE is a nonlinear partial-differential equation that describes evolution of the normalized cavity field envelope $\psi$ over a slow time $\tau=t/2\tau_\gamma$ in a frame parametrized by the ring's azimuthal angle $\theta$ (running from $-\pi$ to $\pi$) co-moving at the group velocity at the frequency of the pump laser. The equation as formulated by Chembo and Menyuk, as it will be used throughout this thesis, reads:
\begin{equation}
\frac{\partial \psi}{\partial \tau}=-(1+i \alpha) \psi + i|\psi|^2 \psi -i \frac{\beta}{2} \frac{\partial^2 \psi}{\partial \theta^2} +F. \label{eq:LLE}
\end{equation}

This equation describes $\psi$ over the domain $-\pi\leq\theta\leq+\pi$ with periodic boundary conditions $\psi(-\pi,\tau)=\psi(\pi,\tau)$. Here $F$ is the normalized strength of the pump laser, with $F$ and $\psi$ both normalized so that they  take the value 1 at the absolute threshold for cascaded four-wave mixing: $F=\sqrt{\frac{8 g_0\Delta\omega_{ext}}{\Delta\omega_{tot}^3}\frac{P_{in}}{\hbar \Omega_0}}$, $|\psi|^2=\frac{2g_0T_{RT}}{\hbar\omega\Delta\omega_{tot}}P_{circ}(\theta,\tau)=\frac{2g_0Ln_g}{c\hbar\omega\Delta\omega_{tot}}P_{circ}(\theta,\tau)$, so that $|\psi(\theta,\tau)|^2$ is the instantaneous normalized power at the co-moving azimuthal angle $\theta$. Here $g_0=n_2 c \hbar \Omega_0^2/(n_g^2 V_0)$ is a parameter describing the four-wave mixing gain, $\Delta\omega_{ext}$ is the rate of coupling at the input/output port, $\Delta\omega_{tot}=1/\tau_\gamma$ is the FWHM resonance linewidth, $P_{in}$ is the pump-laser power, $P_{circ}$ is the circulating power in the cavity, $\hbar$ is Planck's constant, and $\Omega_0$ is the pump-laser frequency. The parameters $n_2$, $n_g$, and $V_0$ describe the nonlinear (Kerr) index (see Eqn. \ref{eq:KerrIndex}), the group index of the mode, and the effective nonlinear mode volume at the pump frequency; $L$ is the physical round-trip length of the ring cavity. 

The parameters $\alpha$ and $\beta$ describe the normalized frequency detuning of the pump laser and second-order dispersion of the resonator mode family into which the pump laser is coupled: $\alpha=-\frac{2(\Omega_0-\omega_{res})}{\Delta\omega_{tot}}$, $\beta=-\frac{2 D_2}{\Delta\omega_{tot}}$; here $D_2=\left.\frac{\partial^2\omega_\mu}{\partial \mu^2}\right|_{\mu=0}$ is the second-order modal dispersion parameter, where $\mu$ is the pump-referenced mode number of Eq. \ref{eq:combfreqsnew}. The parameters $D_1=\left.\frac{\partial\omega_\mu}{\partial\mu}\right|_{\mu=0}=2\pi f_{FSR}$ and $D_2$ are related to the derivatives of the propagation constant $\beta_{prop}=n(\omega)\omega/c$ via $D_1=2\pi/L\beta_1$ and $D_2=-\frac{D_1^2}{\beta_{prop,1}}\beta_{prop,2}$, where $\beta_{prop,n}=\partial^n\beta_{prop}/\partial\omega^n$. The subscript $prop$ is used here to distinguish the propagation constant from the LLE dispersion coefficients $\beta_n=-2D_n/\Delta\omega_{tot}$, as unfortunately the use of the sybmol $\beta$ for both of these quantities is standard. Expressions for higher-order modal dispersion parameters $D_n$ in terms of the expansion of the propagation constant can be obtained by evaluating the equation $D_n=(D_1\frac{\partial}{\partial \mu})^n \omega_\mu$.

The formulation of the LLE in terms of dimensionless normalized parameters helps to elucidate the fundamental properties of the system and facilitates comparison of results obtained in platforms with widely different experimental conditions. In words, the LLE relates the time-evolution of the intracavity field (normalized to its threshold value for cascaded four-wave mixing) to the power of the pump laser (normalized to its value at the threshold for cascaded four-wave mixing), the pump-laser detuning (normalized to half the cavity linewidth), and the cavity second-order disperison quantified by the change in the FSR per mode (normalized to half the cavity linewidth). One example of the utility of this formulation is that it makes apparent the significance of the cavity linewidth in determining the output comb, and underscores the fact that optimization of the dispersion, for example, without paying heed to the effect of this optimization on the cavity linewidth, may not yield the desired results.

The LLE is, of course, a simplified description of the dynamics occurring in the microresonator. It abstracts the nonlinear dynamics and generally successfully describes the various outputs that can be generated in a microresonator frequency comb experiment. The LLE is a good description of these nonlinear dynamics when the resonator photon lifetime, mode-field overlap, and nonlinear index $n_2$ are roughly constant over the bandwidth of the generated comb, and when the dominant contribution to nonlinear dynamics is simply the self-phase modulation term $i|\psi|^2\psi$ arising from the Kerr nonlinearity. The LLE neglects polarization effects, thermal effects, and the Raman scattering and self-steepening nonlinearities, although in principle each of these can be included. It is also worth emphasizing that the LLE can be derived from a more formally accurate Ikeda map (as is done by Coen et al), in which the effect of localized input- and output-coupling is included in the model. This is achieved by `delocalizing' the pump field and the output-coupling over the round trip, including only their averaged effects. This is an approximation that is valid in the limit of high finesse due to the fact that the cavity field cannot change on the timescale of a single round trip, but as a result the LLE necessarily neglects all dynamics that might have some periodicity at the round-trip time; the fundamental timescale of LLE dynamics is the photon lifetime. 

The LLE provides a useful framework for the prediction of comb properties and the exploration of avenues for new research, but also has proven to be an important tool in the interpretation of Kerr-comb experimental results. A primary reason for this is the difficulty of directly characterizing the time-domain output of a microresonator---typically, available tools enable measurement of time-averaged optical spectra and measurement of the comb's output power within the bandwidth of the photodetectors (generally $\lesssim$ 50 GHz) and amplifiers used for the measurement. Since neither records phase information and the latter lacks response on the timescale of, e.g., temporal pulses with duration less than a microresonator's round-trip time, these measurements are insufficient to determine the time-domain output of the microresonator. By providing a mathematical formalism that restricts the possible behaviors of the intracavity field $\psi$, the LLE enables inference of the temporal intensity profile $|\psi(\theta,\tau)|^2$ from the spectrum $|\psi_\mu(\tau)|^2$, which is readily measured in experiment.  \todo{How does FROG fit into this discussion?}


Basically, the LLE predicts the existence of two fundamentally distinct types of Kerr-combs: extended temporal patterns and localized soliton pulses. These predictions are born out by experiments, the interpretation of which is facilitated by insight gained from the LLE. In the remainder of this chapter I briefly present some simple analytical results that can be obtained from the LLE, and then discuss these two types of comb outputs. This discussion provides context for the results presented in the next two chapters. Fig. \ref{fig:paramspace} summarizes the results that will be presented in the remainder of this chapter.
%
%In the following subsections I introduce two fundamentally-distinct types of Kerr-combs. This provides important context for the results presented in the next two chapters. 

\section{LLE analytical curves}
Some insight into comb dynamics can be obtained via analytical investigations of the LLE, Eq. \ref{eq:LLE}. This section largely follows the analysis of Ref. \cite{Godey2014}, with similar analysis having been performed in Refs. \cite{others}. Flat solutions $\psi_s$ may be calculated by setting all derivatives to zero---when these solutions can be realized physically (discussed below), they predict the dependence of the circulating CW power in the resonator, the `CW background,' on the pump power and detuning. Upon setting the derivatives to zero, one finds:
\begin{equation}
F=(1+i\alpha)\psi_s-i|\psi_s|^2\psi_s. \label{eq:LLEstat}
\end{equation}
The circulating intensity $\rho=|\psi_s|^2$ is obtained by taking the modulus-square of Eq. \ref{eq:LLEstat} to get:
\begin{equation}
F^2=\left(1+(\alpha-\rho)^2\right)\rho\label{eq:LLEstat2},
\end{equation} 
and solving this equation for $\rho$. With $\alpha$ held constant, the function $F^2_\alpha(\rho)$ defined by this equation uniquely determines $F^2$ for a given value of $\rho$. By noting that $F^2(\rho=0)=0$ and $\left.\partial F^2/\partial \rho\right|_{\rho=0}>0$, one can conclude that three real solutions for the inverted function $\rho_\alpha(F^2)$ exist between the values of $\rho$ that extremize $F^2_\alpha(\rho)$:
\begin{equation}
\rho_\pm=\frac{2\alpha\pm\sqrt{\alpha^2-3}}{3},
\end{equation}
while outside of this interval there is only one real solution $\rho_\alpha(F^2)$ exists.

Physically, the coexistence of multiple flat solutions $\rho$ at a given point $(\alpha,F^2)$ corresponds to a `tilting' of the Lorentzian transmission profile of the cavity (excluding thermal effects). This is illustrated in Fig. \ref{some figure}. Generally speaking, extended patterns exist on the upper branch of this curve, highlighted in blue, and solitons exist on the lower branch, highlighted in red. For flat solutions $\rho$ having $\partial^2\psi/\partial\theta^2=0$, an effective Kerr-shifted detuning can be defined as $\alpha_{eff}=\alpha-\rho$. The discussion of thermal effects in Sec. \ref{sec:thermaleffects} then applies to the effective detuning $\alpha_{eff}$, which simply incorporates the Kerr nonlinearity into the round-trip phase shift that describes the constructive or destructive interference of the circulating field with the pump at the coupling port. By noting that $\alpha=F^2=\rho$ solves Eq. \ref{eq:LLEstat2}, we can conclude that the position of the effective Kerr-shifted resonance is on the line $\alpha=F^2$, where $\alpha_{eff}=0$. For fixed $F^@$, an effectively red-detuned branch of the tilted resonance exists above the value of $\alpha$ where $\rho$ becomes multivalued. This value of $\alpha$ can be determined by inserting $\rho_-$ (Eq. {\ref{eq:rhopm}) into Eq. \ref{eq:LLEstat2} and solving for $\alpha$. 

Once the circulating intensity $\rho$ is known, the corresponding flat solution $\psi_s$ can be determined from Eq. \ref{eq:LLEstat} by inserting the known value of $\rho$ and solving for $\psi_s$, with the result:
\begin{equation}
\psi_s=\frac{F}{1+i(\alpha-\rho)}.\label{eq:LLEflatsoln}
\end{equation}
This expression reveals that the flat solution acquires a phase $\phi_{CW}=\tan^{-1}(rho-\alpha)$ relative to the pump.


By introducing a perturbation $\delta\psi$ to the flat solution $\psi_s$ and solving for the time evolution of the perturbation, the stability of the flat solution can be investigated---a flat solution $\psi_s$ to the LLE only corresponds to a physically-realizable CW resonator background if it is stable to perturbations. We don't reproduce this analysis in detail here, but state the result:
\begin{itemize}
	\item In the region of multi-stability, if the flat solutions are ordered with increasing magnitude as $\rho_1$, $\rho_2$, and $\rho_3$, the middle solution $\rho_2$ is always unstable. 
	\item A flat solution $\rho$ that is not the middle solution is stable if $\rho<1$; otherwise it is unstable. When the flat solution is unstable, the mode that experiences the greatest instability has mode number $\mu_{max}=\sqrt{\frac{2}{\beta}(\alpha-2\rho)}$.  
\end{itemize}

The pump-laser threshold curve can be determined by setting $\rho=1$ in Eq. \ref{eq:LLEstat}: 
\begin{align}
F^2_{thresh}=1+(\alpha-1)^2, \\
\alpha_{thresh}=1\pm\sqrt{F^2-1},
\end{align} 
for an experiment in which the pump power or detuning is tuned while the other is held fixed. 



\section{Kerr comb outputs: extended modulation-instability patterns}

Extended temporal patterns arise spontaneously as a result of the instability of the flat solution to the LLE when the pump laser is tuned above the threshold curve. These patterns can be stationary, in which case they are typically referred to as `Turing patterns' or `primary comb,' or can evolve in time, in which case they are typically referred to as `noisy comb' or `spatiotemporal chaos.' In general, the former occurs for lower values of the detuning $\alpha$ and smaller pump strengths $F^2$; although some studies of the transition from Turing patterns to chaos have been conducted \cite{Coillet2014,others}, a well-defined boundary between the two has not been established, and may not exist. 

In the spatial domain parametrized by $\theta$, a Turing pattern consists of a periodically modulated waveform with multiple minima and maxima in $\psi^2$. A pulse train with repetition rate $\sim n f_{FSR}$ is coupled out of the resonator. Corresponding to the $n$-fold decreased period (relative to the round-trip time) of an $n$-roll Turing pattern's modulated waveform in the time domain, the optical spectrum of a Turing pattern consists of modes spaced by $n$ resonator FSR---it is this widely-spaced spectrum that is referred to as `primary comb.'  Analytical approximations for Turing patterns are possible near threshold \cite{Lugiato1987,Lugiato1987a} and in the small damping limit \cite{Renninger2016}. The stability analysis results from the last section can be used to predict the spacing $n$ of a primary comb (equivalently the number of Turing-pattern rolls) generated in a decreasing-frequency scan across the resonance with fixed normalized pump power $F^2$: $n=\mu_{max,thresh}=\sqrt{\Delta\omega_0(1+\sqrt{F^2-1})/D_2}$.

Spatiotemporal chaos can be understood as a Turing pattern whose rolls oscillate in height, with adjacent rolls oscillating out of phase. From such an oscillating Turing pattern, if $\alpha$ and/or $F^2$ is increased, one moves deeper into the chaotic regime and pulses begin to exhibit lateral motion and collisions; the number of rolls present in the cavity is no longer constant in time. Depending on the severity of the chaos (higher for larger $\alpha$ and $F^2$), a chaotic comb may correspond to a primary-comb-type spectrum with each primary-comb mode exhibiting sidebands at the resonator FSR, so-called `subcombs,' or it may correspond to a densely-populated spectrum with light in each cavity mode.

Relative to generation of solitons, experimental generation of an extended pattern is straightforward. As shown in Fig. \ref{fig:paramspace}, these patterns are generated with blue pump-laser detuning $\alpha<0$, where thermal locking occurs. Because they arise spontaneously from noise, their generation is straightforward: simply decrease the pump-laser frequency until a pattern is generated. Unfortunately, operation of a Kerr-comb in the extended pattern regime is disadvantageous for applications: the $n$-FSR spacing of primary comb presents a challenge for measurement of the repetition rate of the frequency comb due to the bandwidth of measurement electronics, and the aperiodic time-evolution of chaotic comb corresponds to modulation sidebands on the comb modes within the linewidth of the cavity that preclude the use of the comb as a set of stable optical reference frequencies. 

An important property of these extended patterns is that they fill the resonator---while the characteristic size of temporal features scales roughly as $1/\sqrt{-\beta}$, these features are distributed densely and uniformly throughout the resonator. This means that the total circulating power of an extended pattern $\int d\theta\, |\psi|^2$ is large relative to the localized pulses discussed in the next section, and therefore that extended powers comb with a comparatively large thermal shift of the resonance. 



\section{Kerr comb outputs: solitons} \label{sec:LLEsolitons}

The term `soliton' generally refers to a localized excitation that can propagate without changing its shape due to a delicate balance between dispersion (or diffraction) and nonlinearity. Solitons are found in several contexts within the field of nonlinear optics, and temporal Kerr soliton pulses in optical fibers are particularly well known. Microresonators support so-called dissipative cavity solitons, which are localized pulses circulating the resonator that are out-coupled once per round trip. In the case of a single circulating soliton, this leads to a train of pulses propagating away from the resonator with repetition rate $1/T_{RT}$. Thus the mode spacing of the comb matches the FSR of the resonator, in contrast with widely-spaced primary comb spectra, and the soliton can, in principle, propagate indefinitely as a stationary solution to the LLE. This makes Kerr combs based on solitons particularly attractive for applications.

Solitons in optical fibers are solutions of the nonlinear Schrodinger equation (NLSE) that describes pulse-propagation in optical fiber \cite{Agrawal2007}:
\begin{equation}
\frac{\partial A}{\partial z}= i\gamma|A|^2 A -i \frac{\beta}{2} \frac{\partial^2 A}{\partial T^2}. \label{NLSE}
\end{equation}
This equation describes the evolution of the pulse envelope $A$ in the `fast-time' reference frame parametrized by $T$ as it propagates down the length of the fiber, parametrized by the distance variable $z$. Here $\gamma$ is the nonlinear coefficient of the fiber and $\beta$ is the GVD parameter\todo{reference to the above}. The LLE can be viewed as an NLSE with additional loss and detuning terms $-(1+i\alpha)\psi$ and a driving term $F$.

The fundamental soliton solution to the NLSE is:
\begin{equation}
A_{sol}=\sqrt{P_0}\, \mathrm{sech}\left(T/\tau\right)\,e^{i\gamma P_0 z/2+i\phi_0},
\end{equation}
where $P_0$ is the peak power of the pulse and is related to the duration of the pulse $\tau$ via $\tau=\sqrt{-\beta/\gamma P_0}$, and $\phi_0$ is an arbitrary phase. Thus, this equation admits a \textit{continuum} of pulsed fundamental `soliton' solutions, with one existing for each value of the peak power. Each of these solutions propagates down the fiber without changing shape; only the phase evolves with distance as $\phi(z)=\gamma P_0 z/2+\phi_0$.


The introduction of the loss, detuning, and driving terms into the NLSE to obtain the LLE has several important consequences for solitons. First, exact analytical expressions for the soliton solution to the LLE in terms of elementary functions are not known, in contrast with the situation for the NLSE. However, the soliton solutions to the LLE, Eq. \ref{eq:LLE}, can be approximated well as:
\begin{equation}
\psi_{sol}=\psi_{s,min}+e^{i\phi_0}\sqrt{2\alpha}\,\mathrm{sech}\sqrt{\frac{2\alpha}{-\beta}}\theta. \label{eq:LLEsoliton}
\end{equation}
Here $\psi_{s,min}$ is the flat solution to the LLE from Eq. \ref{eq:LLEflatsoln} at the point where the soliton solution is desired; when multiple flat solutions exist, $\psi_{s,min}$ is the one corresponding to the smallest intensity $\rho_1$. The phase $\phi_0=\mathrm{cos}^{-1}(\sqrt{8\alpha}/\pi F)$ arises from the intensity-dependent phase shift in the cavity due to the Kerr effect, mathematically described by the term $i|\psi|^2\psi$. 

This approximation $\psi_{sol}$ from Eq. \ref{eq:LLEsoliton} for the soliton solution of the LLE illustrates a second important consequence of the differences between the NLSE and the LLE: while the NLSE admits a continuum of fundamental soliton solutions parametrized by their peak power $P_0$ and arbitrary phase $\phi_0$, the LLE supports only one shape for the envelope of a soliton for fixed experimental parameters $\alpha$ and $F^2$. Intuitively, this can be understood as arising from the need for a balance between dispersion and nonlinearity, as in the NLSE, \textit{and} between loss (described by $\partial\psi/\partial\tau=-\psi+...$) and the pump (described by $\partial\psi/\partial\tau=...+F$) for stable evolution of an LLE soliton---the driving term is not scaled by $\psi$, which instead would represent linear gain, and therefore provides an absolute reference that fixes the amplitude of the soliton.

The amplitude of the LLE soliton depends only on the detuning $\alpha$, and the width of the soliton increases with larger detuning $\alpha$ and smaller dispersion $\beta$. These characteristics are apparent from the analytical approximation in Eq. \ref{eq:LLEsol}, but are also evident in numerical calculations of the exact soliton solution to the LLE\cite{Yi2015}\todo{check if this is an appropriate reference}.

Solitons exist only where there is a flat solution $\rho$ that is effectively red detuned $\alpha_{eff}=\alpha-\rho>0$ that can form the background $\psi_s$ for the pulse\cite{Barashenkov1996,Coen2013}. \todo{PRE 54, 5707 1996}. Consistent with the phase $\phi_0$ in the approximation $\psi_{sol}$ in Eq. \ref{eq:LLEsoliton}, solitons can exist up to a maximum detuning of $\alpha_{max}=\pi^2 F^2/8$\cite{Herr2014}.

%is clearly valid only for red pump-laser detuning  Solitons exist as solutions to the LLE only for red pump-laser detuning $\alpha>0$.

Solitons are strongly localized: the deviation of the background intensity from $\rho_1$ near a soliton at $\theta_0$ is proportional to $e^{-(\theta-\theta_0)/\delta\theta}$, where $\delta\theta=\sqrt{-\beta/2\alpha}$. If $\delta\theta$ is sufficiently small, multiple solitons can be supported in the resonator domain $-\pi\leq\theta\leq\pi$ with negligible interactions between solitons. Simulations reveal that if $(\theta-\theta_0)/\delta\theta$ is too small, solitons exhibit attractive interactions\todo{should I add some references here?}; the result of this attraction can be pair-wise annihilation or pair-wise merger, with the ultimate result being a stable soliton ensemble with fewer solitons. The maximum number of solitons that can coexist in a resonator in the absence of higher-order stabilizing effects (see Chapter \ref{ch:solitoncrystals}) can be approximated as $N_{max}\approx\sqrt{-2/\beta}$.

A single soliton is a localized pulse that is repeated once each round-trip; therefore a single soliton is a single-FSR frequency comb, in contrast with the primary comb output discussed above. The spectrum of a single-soliton comb has a $\mathrm{sech}^2(\Omega_0/\Delta \Omega)$ envelope, where $\Delta\Omega\approx\sqrt{32\alpha/|\beta| T_{RT}^2 }$ is the bandwidth of the pulse in angular frequency. Equivalently, the bandwidth of the soliton in (linear) optical frequency is $\sqrt{\frac{16\Delta\nu f_{rep}^2}{D_2}\alpha}$. For a soliton at the maximum detuning $\alpha_{max}=\pi^2F^2/8$ for fixed normalized pump power $F^2$, the bandwidth is then $\sqrt{\frac{\pi^2\Delta\nu f_{rep}^2}{2 D_2}F^2}$. Because solitons have single-FSR spacing, have the output localized into a high peak-power pulse, and are stationary (in contrast with chaos, which has single-FSR spacing but is not-stationary), they are useful for applications. Many of the proposals for and demonstrations of applications with Kerr-combs have used single-soliton operation. 

\subsection{Experimental generation of solitons}


Relative to the generation of extended modulation-instability patterns, experimental generation of solitons in microring resonators is challenging. Solitons are localized excitations below threshold, which means that their existence is degenerate with their absence---a resonator can host $N=0, 1, 2,...$ up to $N_{max}$ solitons for a given set of parameters $\alpha$ and $F^2$; this degeneracy is illustrated in Fig. \ref{fig:energylevels}. If $\alpha$ and $F^2$ are experimentally tuned to a point at which solitons may exist, $\psi$ will evolve to a form determined by the initial conditions. To provide appropriate initial conditions, most experimental demonstrations of soliton generation have involved first generating an extended pattern in the resonator, and then tuning to an appropriate point $(\alpha,F^2)$ so that `condensation' of solitons from the extended pattern occurs. 

Condensation of solitons from an extended pattern presents additional challenges. First, it is difficult to control the number of solitons that emerge, due to the high degree of soliton-number degeneracy. This challenge typically leads to a success rate somewhat lower than 100 $\%$ in the generation of single solitons. Second, the transition from a high duty-cycle extended pattern to a lower duty-cycle ensemble of one or several solitons comes with a dramatic drop in intracavity power that occurs on the timescale of the photon lifetime. If the resonator is in thermal steady-state before this drop occurs, the resonator will cool and the resonance frequency will increase. This increase is usually large enough that the final detuning $\alpha$ exceeds $\alpha_{max}=\pi^2 F^2/8$ for solitons, and the soliton is lost. This challenge can be addressed by preparing initial conditions for soliton generation and then tuning to an appropriate point $(\alpha,F^2$) before the cavity can come into thermal steady-state at the temperature determined by the large power of the extended pattern; this is possible because the timescale over which an extended pattern can be generated is related to the photon lifetime, which is typically much faster than the thermal timescale.

The first report of soliton generation in microresonators came in a paper by Herr et al published in 2014\cite{Herr2014}. These authors described optimizing the speed of a decreasing-frequency scan of the pump laser across the cavity resonance so that a soliton could be condensed from an extended pattern and the scan could then be halted at a laser frequency where the soliton could be maintained, with the system in thermal steady-state at the temperature determined by the circulating power of the soliton. Other approaches for dealing with the challenges described above have been developed since this first demonstration; these include fast manipulation of the pump power \cite{Brash2015,Yi2015} and periodic modulation of the pump laser's phase or power at $f_{FSR}$\cite{Lobanov2015,Obrzud2017}\todo{do I have the right version of Obrzud?}. These methods continue to make use of extended patterns to provide initial conditions for soliton generation. In formally-equivalent fiber-ring resonators, direct generation of solitons without condensation from an extended pattern has been demonstrated using transient phase and/or amplitude modulation of the pump laser \cite{Jang1,Jang2,AMpaper}. Chapter \ref{chap:PMpumping} presents a new variation on these schemes that enables direct generation of solitons using only phase modulation at $f_{FSR}$ without transient manipulation of the system parameters; this approach is based on a proposal by Taheri et al \cite{Taheri2016}. 

A variety of applications of soliton-based Kerr frequency combs have already been demonstrated. 

incorporate:
Ferdous 2011 spectral line-by-line shaping



%\section{Frequency comb nonlinear dynamics in microresonators}
%
%The first report of frequency-comb generation in silica microcavities by Del'Haye et al initiated a serious research effort to understand how to make these combs useful for applications. The early years of this effort brought reports of comb generation in resonators made from amorphous \cite{stuff} and crystalline \cite{stuff} materials, with 



%
%
%
%
%
%
%that exhibit the  nonlinearity, which can be the dominant source of nonlinearity in centro-symmetric materials in which the second-order nonlinearity $\chi{(2)}$ must vanish. The $\chi^{(3)}$ nonlinearity, or Kerr nonlinearity, arises from the term in the Taylor expansion for the polarization response of the medium that scales with the optical intensity: $P_{NL}=...+\epsilon_0 \chi^{(3)} E^3+...$. \cite{something}. This can equivalently and usefully be described as a dependence of the refractive index on the intensity via .
%
%The third-order 
%
%The resonator quality factor is an important figure of merit for the use of optical resonators as platforms for nonlinear optics. This is discussed extensively below; typically the threshold power for nonlinear optics scales as $Q^{-2}$, meaning that in the design of practical platforms targeting high $Q$ is an important consideration. Fabrication of ultrahigh-Q resonators has been achieved with a variety of designs and materials, including ... 
%
%
% These resonators can be constructed with extremely high quality factors, upwards of $10^8$, which facilitates high circulating intensity. 
% 
% Combs using optical ring resonators leverage this confinement and the resulting long photon lifetimes in high-quali oty factor resonators 