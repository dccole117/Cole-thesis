\chapter{Microresonators}
 \label{ch:microresonators}
 
 This chapter introduces the basic physics of Kerr-nonlinear optical ring resonators, and the two \todo{correct?} subsequent chapters describe results obtained in these systems.
 
An optical ring resonator, shown schematically in Fig. \ref{fig:RingResonator}, guides light around a closed path in a dielectric medium by total internal reflection, similar to the mechanism that guides light in an optical fiber. The utility of these resonators lies in the fact that light can circulate for many roundtrips before it is coupled out or dissipated, which means that very high circulating intensities can be achieved.

A ring resonator supports propagating guided optical \textit{modes} of electromagnetic radiation that occur at (vacuum) wavelengths that evenly divide the optical round-trip path length: $\lambda_m=n_{eff}(\lambda_m)L/m$, with associated resonance frequencies $\nu_m=c/\lambda_m=mc/n_{eff}(\nu_m)L$, leading to constructive interference from round-trip to round-trip. Here $L$ is the physical round-trip length of the resonator, $m$ is the azimuthal mode number, and $n_{eff}(\lambda_m)$ is an effective index of refraction that depends on the resonator geometry and the mode's transverse mode profile (see e.g. \cite{REFHERE} for more information). The free-spectral range $f_{FSR}$ of a resonator is the \textit{local} frequency spacing between modes, calculated via:
\begin{align}
	f_{FSR}&\approx \nu_{m+1}-\nu_{m}\approx \nu_{m}-\nu_{m-1},\\
	&=\frac{\partial\nu_m}{\partial m},\\
	&=\frac{c}{n_{eff}(\nu)L}-\frac{mc}{n_{eff}^2(\nu)L}\frac{\partial n_{eff}}{\partial \nu}\frac{\partial \nu}{\partial m},\\
	\Rightarrow f_{FSR}&=\frac{c/L}{\left(n_{eff}+\nu\frac{\partial n_{eff}}{\partial \nu}\right)}=\frac{c}{n_g L}=1/T_{RT},
\end{align}
	where $n_g=n_{eff}+\nu\frac{\partial n_{eff}}{\partial \nu}$ is the group velocity of the mode and $T_{RT}$ is the mode's round-trip time. Importantly, both intrinsic material dispersion and geometric dispersion resulting from, e.g., different sampling of core versus cladding material properties for different transverse mode profiles, lead to a frequency dependence for each of the parameters $n_{eff}$, $n_g$, and $f_{FSR}$, and a resulting non-uniform spacing in the cavity modes in frequency despite the linearity of $\nu_m$ in $m$. 
	
	 Unless special efforts are made, ring resonators are typically multi-mode, meaning that many different transverse mode profiles are supported. To calculate the frequency-dependent effective index $n_{eff}(\nu)$, thereby enabling calculation of the resonance frequencies and wavelengths, one must solve Maxwell's equations for the resonator geometry. Except in special cases of high symmetry \cite{microsphereresonators}, this is typically done numerically using finite-element modeling tools like COMSOL. The modes of an optical resonator, both within a mode family defined by a transverse mode profile and between mode families, must be orthogonal\todo{is there a good citation here?}. 

 The lifetime $\tau_\gamma$ of circulating photons in a resonator is fundamental to its fitness for applications. Generally, two processes lead to the loss of circulating photons: intrinsic dissipation that occurs at a rate $1/\tau_{int}$ and outcoupling to an external waveguide that occurs at a rate $1/\tau_{ext}$, leading to a total loss rate of $\tau_\gamma^{-1}=\tau_{ext}^{-1}+\tau_{int}^{-1}$. To understand the quantitative role of these parameters, we consider a cavity mode of frequency $\omega_0$ and amplitude $a$ (normalized such that $|a|^2=N$, the number of circulating photons) driven by a field with frequency $\omega$ and rotating amplitude $s\propto\exp(i\omega t)$ (normalized such that $|s|^2=S$, the rate at which photons in the coupling waveguide pass the coupling port) that is in-coupled with strength $\kappa$. The equation of motion for such a system is\cite{Haus1984}:
 \begin{equation}
 \frac{d a}{d t}=i\omega_0 a-\left(\frac{1}{2\tau_{int}}+\frac{1}{2\tau_{ext}}\right)a+\kappa s. \label{eq:coupledmotion}
 \end{equation}
 We can immediately solve this equation by assuming that $a\propto\exp(i\omega t)$, and we obtain:
 \begin{equation}
 a=\frac{\kappa s}{\left(\frac{1}{2\tau_{int}}+\frac{1}{2\tau_{ext}}\right)+i(\omega-\omega_0)}. \label{eq:coupledsoln}
 \end{equation}
 
 To extract anything further from this equation, we must derive a relationship between $\tau_{ext}$ and $\kappa$, which so far are not related. To do this, we exploit the time-reversal symmetry that is inherent in this system when there is no dissipation, that is, when $1/\tau_{int}=0$. In the case of only an initial excitation $N_0$ decaying into the waveguide with the driving term set to zero, we have $N=N_0e^{-t/\tau_{ext}}$. In this case, energy conservation guarantees that the rate $S_out$ at which photons propagate away from the resonator in the waveguide is $-dN/dt=N_0e^{-t/\tau_{ext}}/\tau_{ext}$; we therefore have $S_{out}=N/\tau_{ext}$. In the time-reversed system with $t\rightarrow-t$, this amplitude $S_{out}$ describes the rate of pumping: the cavity is resonantly driven with increasing power $S=S_{out}(-t)=N_0\exp(t/\tau_{ext})/\tau_{ext}$. In this case the frequency of the driving field $s$ can be written $\omega_0-i/2\tau_{ext}$. Inserting this frequency into Eq. \ref{eq:coupledsoln} gives the equations
 \begin{align}
  a&=\kappa s \tau_{ext}\, \text{and}\\
   N&=|\kappa|^2 S \tau_{ext}^2.
   \end{align}
   By comparing the relationships between $S_{out}$ and $N$ for the forward-evolving system and $S$ and $N$ for the backward-evolving system, we arrive at the relationship $|\kappa|^2=1/\tau_{ext}$. This relationship holds generally, and we can return to the case including dissipation and insert it into Eq. \ref{eq:coupledsoln}, which can then be squared to obtain:
   \begin{equation}
   N=\frac{\Delta\omega_{ext}S}{\Delta\omega_{tot}^2/4+(\omega-\omega_0)^2} \label{eq:resenhancement}
   \end{equation}
   Here we define the linewidths $\Delta\omega_{ext}=1/\tau_{ext}$, $\Delta\omega_{int}=1/\tau_{int}$, and $\Delta\omega_{tot}=\Delta\omega_{ext}+\Delta\omega_{int}$. Two important observations can be drawn from Eq. \ref{eq:resenhancement}: First, the cavity response is Lorentzian with a full-width at half-maximum (FWHM) linewidth that is related to the photon lifetime via $\tau_\gamma=1/\Delta\omega_{tot}$, and second, on resonance the number of circulating photons is related to the input rate by the factor $\Delta\omega_{ext}/\Delta\omega_{tot}^2$. The combination of this resonant enhancement and a small cavity mode volume enables very large circulating optical intensities, which is important for the application of microresonators in nonlinear optics.
   
   Two commonly used practical quantities are linked to the photon lifetime: the resonator finesse $\mathcal{F}=2\pi\tau_\gamma/T_{RT}$, which for a ring resonator can be interpreted literally as the azimuthal resonator angle traced out by a typical photon over its lifetime; and the resonator quality factor $Q=\omega_c \tau_\gamma$, the phase over which the optical field evolves during the photon lifetime. Using the relationship $\tau_\gamma=1/\Delta\omega_{tot}$, the finesse and quality factor can be rewritten as simple ratios of the relevant frequencies: $\mathcal{F}=f_{FSR}/\Delta\nu$; $Q=\nu_c/\Delta\nu$, where $\Delta\nu=\Delta\omega_{tot}/2\pi$.
   
   
  
 
 
% 
% 
% Energy conservation ensures that the rate at which photons propagate away from the resonator in the coupling waveguide is $|s|^2=S=-dN/d t=\frac{N_0}{\tau_{ext}}\exp(-t/\tau_{ext})$. We can consider the time-reversed system in which the input increases in magnitude as $S=\frac{N_0}{\tau_{ext}}\exp(t/\tau_{ext})$ and is resonant with the cavity mode. In this case, the frequency of the drive $s$ is $\omega_s=\omega_0-i/2\tau_{ext}$. Inserting this frequency into Eq. \ref{eq:coupledsoln} for the case of $\Delta\omega_{int}=0$ yields:
% However, for the system without time reversal we have $N=N_0\exp(-t/\tau_{ext})$, $S=\frac{N_0}{\tau_{ext}}\exp(-t/\tau_{ext})$, so that $N=\tau_{ext} S$. By inserting this into the previous equation, we obtain . 
% This is quantified by the basic relation for the number of circulating photons $N(t)=N_oe^{-t/\tau_\gamma}$ in the presence of solely linear loss, which defines the photon lifetime $\tau_\gamma$. 
%
%\begin{equation}
%\mathcal{F}\{E\}(\omega)\propto\int_0^\infty dt\, e^{-\left(\frac{1}{2\tau_\gamma}+i(\omega_c-\omega)\right)t},
%\end{equation}
%which immediately yields the Lorentzian lineshape
%\begin{equation}
%|\mathcal{F}\{E\}|^2\propto\frac{1}{(\omega-\omega_c)^2+\frac{1}{4\tau_\gamma^2}}, \label{eq:lorentzian}
%\end{equation}
%with FWHM linewidth $\Delta\omega=1/\tau_\gamma$. With this relationship, 
%
%The utility of a resonator with long photon lifetime is illustrated by calculating the steady-state number of circulating photons $N$ in a system including a driving term $S$ denoting the rate at which photons are added\cite{Haus1984}. Working in the field quantities $a$ ($|a|^2=N$) and $s$ ($|s|^2=S$), the equation of motion for such a system is: 
%
%The terms in this equation represent the rotation of the field at the mode's frequency $\omega_0$, the strength $\kappa$ of coupling from the input waveguide to the resonator, and the intrinsic and extrinsic (power) loss timescales $\tau_{int}=1/\Delta\omega_{int}$ and $\tau_{ext}=1/\Delta\omega_{ext}$ due to loss and outcoupling. If the drive has frequency $\omega_s$, 
%
%The rates $\Delta\omega_{ext}$ and $\kappa$ are not \textit{a priori} related, but we can derive the relationship between them by exploiting the time-reversal symmetry of systems without dissipation. For the case of no pumping ($s=0$) and no losses ($\Delta\omega_{int}=0$), we have 
%\begin{align}
%\frac{d a}{dt}&=i\omega_0 a-\frac{a}{2\tau_{ext}},\\
%\Rightarrow &a=a_0\exp(i\omega_0 t-t/2\tau_{ext}), \\
%&N=N_0\exp(-t/\tau_{ext}).
%\end{align}





%For the case without dissipation, then, we know that $N=S\tau_ext$, which is the solution of the rate equation
%\begin{equation}
%\frac{dN}{dt}=-N
%\end{equation}
%
%We can now derive a rate equation describing $N(t)$ from Eq. \ref{eq:coupledmotion}:
%\begin{align}
%\frac{dN}{dt}=\frac{d|a|^2}{dt}=a^*\frac{da}{dt}+a\frac{da^*}{dt}
%\end{align}





%Thus the steady-state number of circulating photons and therefore the intensity is larger than the input rate by a factor of the photon lifetime. Of course a more complete calculation including effects such as outcoupling at the coupler (with a rate related to the driving term $A$) could be performed, but this simple calculation captures the essence of the importance of photon lifetime $\tau_\gamma$. \todo{I am not sure whether this captures the scaling that I want to indicate}

\section{Basic experiments}

In a typical microresonator frequency-comb experiment, a frequency-tunable pump laser is coupled evanescently into and out of the resonator using a tapered optical fiber (for e.g. free-standing silica disc resonators) or a bus waveguide (for chip-integrated resonators, e.g. in silicon nitride rings). Throughout this thesis, the wavelength of the pump laser is always in the telecommunications band, near $\lambda=$1550 nm. However, other pump wavelengths are possible, and frequency combs have been generated with pumps ranging from the visible \cite{visiblecombs} to the outer reaches of the near infrared \cite{midIRcombs}\todo{true?}. When overlap between the evanescent mode of the fiber and a whispering-gallery mode of the resonator is achieved, with the frequency of the pump laser close to the resonant frequency of that mode, light will build up in the resonator and the transmission of the pump laser past the resonator will decrease.

In any experiment in which a significant amount of pump light is coupled into a resonator, one immediately observe that the cavity resonance lineshape in a scan of the pump-laser freuqency is not Lorentzian as expected from Eq. \ref{eq:lorentzian}. This is due to heating of the resonator as it absorbs circulating optical power. Since the mode-field volume and the physical volume of the microresonator are both small, thermal effects are large enough to have important practical implications in microresonator experiments. As the local volume of the mode heats (over the so-called `fast thermal time-scale') and this energy is conducted to and heats the rest of the resonator (over the `slow thermal time-scale') \cite{distinctThermalTimescales}, the resonance frequency of a given cavity mode shifts due to the thermo-optic coefficient $\partial n/\partial T$ and the coefficient of thermal expansion of the mode volume $\partial V/\partial T$. For typical microresonator materials the thermo-optic effect dominates, and $\partial n/\partial T>0$ leads to a decrease in the resonance frequency with increased circulating power.

A calculation of the thermal dynamics of the system composed of the pump laser and the resonator reveals that there is a range of pump-laser frequencies (which depends on the pump laser power) near and below the `cold-cavity' resonance frequency of a given cavity mode over which the system has three possible thermally-shifted resonance frequencies at which thermal steady-state is achieved. Generally, these points are:
\begin{enumerate}
\item $\Omega_o>\omega_{res,shifted}$,
\item $\Omega_o<\omega_{res,shifted}$,
\item $\Omega_o\ll\omega_{res,shifted}$,
\end{enumerate}
where $\omega_{res,shifted}$ is the resonance frequency of the cavity mode including thermal effects. These points correspond to the case of (1) Blue detuning with significant coupled power and a thermal shift of the resonance, (2) Red detuning with significant coupled power and a thermal shift of the resonance, and (3) Large red detuning with no significant coupled power and no thermal shift of the resonance. Steady-state point (1) is experimentally important, because in the presence of pump-laser frequency and power fluctuations it leads to so-called thermal `self-locking.' Specifically for steady-state point (1), this can be seen as follows: 
\begin{itemize}
	\item If the pump-laser power increases, the cavity heats, the resonance frequency decreases, the detuning increases, and the change in coupled power is minimized.
	\item If the pump-laser power decreases the cavity cools, the resonance frequency increases, the detuning decreases, and the change in coupled power is minimized.
	\item If the pump-laser frequency increases the cavity cools, the resonance frequency increases, and the change in detuning is minimized.
	\item If the pump-laesr frequency decreases the cavity heats, the resonance frequency decreases, and the change in detuning is minimized.
	\end{itemize}
This is in contrast with steady-state point (2), where each of the four pump-laser fluctuations considered above generates a positive feedback loop, with the result that any fluctuation will lead the system to flip to steady-state point (1) or (3). This preference of the system to occupy point (1) or point (3) over a range of red detuning is referred to as thermal bi-stability. One consequence is that the transmission profile of the pump laser takes on hysteretic behavior in a scan over a cavity resonance with significant pump power: in a decreasing frequency scan, the lineshape takes on a broad sawtooth shape, while in an increasing frequency scan, the resonance takes on a narrow pseudo-Lorentzian profile whose exact shape depends on the scan parameters relative to the thermal timescale. A second consequence is that operation at red detuning with significant coupled power in a microresonator experiment requires special efforts to mitigate the effects of thermal instability.








\section{Microring resonator Kerr frequency combs}

The high circulating optical intensities accessible in resonators with long photon lifetimes find immediate application in the use of microresonators for nonlinear optics. The experiments described in this thesis are conducted in silica microresonators. Silica falls into a broader class of materials that exhibit both centro-symmetry, which dictates that the second-order nonlinear susceptibility $\chi^{(2)}$ must vanish, and a significant third-order susceptibility $\chi^{(3)}$. The $n$\textsuperscript{th}-order susceptibility is a term in the Taylor expansion describing the response of the medium's polarization to an external electric field: $P=P_0+\epsilon_0 \chi^{(1)} E + \epsilon_0 \chi^{(2)} E^2 + \epsilon_0 \chi^{(3)} E^3+...$. The effect of $\chi^{(3)}$ can be described in a straightforward way as a dependence of the refractive index on the local intensity\cite{somethingelse},
\begin{equation}
n=n_0+n_2 I \label{eq:KerrIndex}
\end{equation}
where $n_2=\frac{3\chi^{(3)}}{4n_0^2\epsilon^0 c}$\cite{CosoAndSolis,alsoLLbook?}. The intensity-dependence of the refractive index is referred to as the optical Kerr effect.

The combination of the Kerr effect and the high circulating intensities that are accessible in high-finesse cavities provides a powerful platform for nonlinear optics. Specifically, the Kerr effect (or third-order susceptibility $\chi^{(3)}$) enables self-phase modulation, cross-phase modulation, and four-wave mixing, the last of which is depicted schematically in Fig. \ref{fig:FWM}. 

In 2007, a remarkable result heralded the beginning of a new era for frequency comb research. Del'Haye et al reported \textit{cascaded four-wave mixing} in toroidal silica microcavities on silicon chips, the result of which was a series of many co-circulating optical fields that were uniformly spaced by $f_{rep}$ ranging from 375 GHz to $\sim$750 GHz (depending on the platform)\cite{DelHaye2007}. Measurements indicated that the frequency spacing was uniform to a precision of $7.3 \times 10^{-18}$, thereby establishing that the output of the system was a frequency comb. This result built on previous demonstrations of few-mode parametric oscillations in microresonators \cite{Kippenberg2004, Savchenkov2004,Agha2007}\todo{from initial kerr comb paper}, and showed that the non-uniform distribution of cavity resonance frequencies could be overcome to generate an output with equidistant frequency modes. Demonstrations of frequency-comb generation in other platforms followed shortly, with realizations in ... A second important development occurred in 2012, when Herr et al reported the generation of frequency combs corresponding in the time domain to single circulating optical `soliton' pulses. In fact, it is now recognized that passive fiber ring resonators are formally equivalent to Kerr microrings, and solitons were generated in these systems in 2010\cite{Leo2010}. 

Kerr-comb generation can be motivated and partially understood through the CFWM picture, but the phase and amplitude degrees of freedom for each comb line mean that CFWM gives rise to rich space of comb phenomena, as evidenced by the various behaviors reported in Kerr microresonators. A useful model for understanding this rich space is the Lugiato-Lefever equation (LLE), which was shown to describe microcomb dynamics by Chembo and Menyuk \cite{Chembo2013} through Fourier-transformation of a set of coupled-mode equations describing CFWM and by Coen et al \cite{Coen2013} through time-averaging of an Ikeda map for a low-loss resonator (as first performed by Haelterman, Trillo, and Wabnitz \cite{Haelterman1992}).  The LLE is a nonlinear partial-differential equation that describes evolution of the normalized cavity field $\psi$ over a slow time $\tau=t/2\tau_\gamma$ in a frame parametrized by the ring's azimuthal angle $\theta$ (running from $-\pi$ to $\pi$) co-moving at the group velocity at the frequency of the pump laser. The equation as formulated by Chembo and Menyuk, as it will be used throughout this thesis, reads:
\begin{equation}
\frac{\partial \psi}{\partial \tau}=-(1+i \alpha) \psi + i|\psi|^2 \psi -i \frac{\beta}{2} \frac{\partial^2 \psi}{\partial \theta^2} +F.
\end{equation}

This equation describes $\psi$ over the domain $-\pi\leq\theta\leq+\pi$ with periodic boundary conditions $\psi(-\pi,\tau)=\psi(\pi,\tau)$. Here $F$ is the normalized strength of the pump laser, with $F$ and $\psi$ both normalized so that they  take the value 1 at the absolute threshold for cascaded four-wave mixing: $F=\sqrt{\frac{8 g_0\Delta\omega_{ext}}{\Delta\omega_{tot}^3}\frac{P_{in}}{\hbar \Omega_0}}$, $|\psi|^2=\frac{2g_0T_{RT}}{\hbar\omega\Delta\omega_{tot}}P_{circ}(\theta,\tau)=\frac{2g_0Ln_g}{c\hbar\omega\Delta\omega_{tot}}P_{circ}(\theta,\tau)$, so that $|\psi(\theta,\tau)|^2$ is the instantaneous normalized power at the co-moving azimuthal angle $\theta$. Here $g_0=n_2 c \hbar \Omega_0^2/(n_g^2 V_0)$ is a parameter describing the four-wave mixing gain, $\Delta\omega_{ext}$ is the rate of coupling at the input/output port, $\Delta\omega_{tot}=1/\tau_\gamma$ is the FWHM resonance linewidth, $P_{in}$ is the pump-laser power, $P_{circ}$ is the circulating power in the cavity, $\hbar$ is Planck's constant, and $\Omega_0$ is the pump-laser frequency. The parameters $n_2$, $n_g$, and $V_0$ describe the nonlinear (Kerr) index (see Eqn. \ref{eq:KerrIndex}), the group index of the mode, and the effective nonlinear mode volume at the pump frequency; $L$ is the physical round-trip length of the ring cavity. 

The parameters $\alpha$ and $\beta$ describe the normalized frequency detuning of the pump laser and second-order dispersion of the resonator mode family into which the pump laser is coupled: $\alpha=-\frac{2(\Omega_0-\omega_{res})}{\Delta\omega_{tot}}$, $\beta=-\frac{2 D_2}{\Delta\omega_{tot}}$; here $D_2=\left.\frac{\partial^2\omega_\mu}{\partial \mu^2}\right|_{\mu=0}$ is the second-order modal dispersion parameter, where $\mu$ is the pump-referenced mode number of Eq. \ref{eq:combfreqsnew}. The parameters $D_1=\left.\frac{\partial\omega_\mu}{\partial\mu}\right|_{\mu=0}=2\pi f_{FSR}$ and $D_2$ are related to the derivatives of the propagation constant $\beta_{prop}=n(\omega)\omega/c$ via $D_1=2\pi/L\beta_1$ and $D_2=-\frac{D_1^2}{\beta_{prop,1}}\beta_{prop,2}$, where $\beta_{prop,n}=\partial^n\beta_{prop}/\partial\omega^n$. The subscript $prop$ is used here to distinguish the propagation constant from the LLE dispersion coefficients $\beta_n=-2D_n/\Delta\omega_{tot}$, as unfortunately the use of the sybmol $\beta$ for both of these quantities is standard. Expressions for higher-order modal dispersion parameters $D_n$ in terms of the expansion of the propagation constant can be obtained by evaluating the equation $D_n=(D_1\frac{\partial}{\partial \mu})^n \omega_\mu$.

The formulation of the LLE in terms of dimensionless normalized parameters helps to elucidate the fundamental properties of the system and facilitates comparison of results obtained in platforms with widely different experimental conditions. In words, the LLE relates the time-evolution of the intracavity field (normalized to its threshold value for cascaded four-wave mixing) to the power of the pump laser (normalized to its value at the threshold for cascaded four-wave mixing), the pump-laser detuning (normalized to half the cavity linewidth), and the cavity second-order disperison quantified by the change in the FSR per mode (normalized to half the cavity linewidth). One example of the utility of this formulation is that it makes apparent the significance of the cavity linewidth in determining the output comb, and underscores the fact that optimization of the dispersion, for example, without paying heed to the effect of this optimization on the cavity linewidth, may not yield the desired results.

The LLE is, of course, a simplified description of the dynamics occurring in the microresonator. It abstracts the nonlinear dynamics and generally successfully describes the various outputs that can be generated in a microresonator frequency comb experiment. The LLE is a good description of these nonlinear dynamics when the resonator photon lifetime, mode-field overlap, and nonlinear index $n_2$ are roughly constant over the bandwidth of the generated comb, and when the dominant contribution to nonlinear dynamics is simply the self-phase modulation term $i|\psi|^2\psi$ arising from the Kerr nonlinearity. The LLE neglects polarization effects, thermal effects, and the Raman scattering and self-steepening nonlinearities, although in principle each of these can be included. It is also worth emphasizing that the LLE can be derived from a more formally accurate Ikeda map (as is done by Coen et al), in which the effect of localized input- and output-coupling is included in the model. This is achieved by `delocalizing' the pump field and the output-coupling over the round trip, including only their averaged effects. This is an approximation that is valid in the limit of high finesse due to the fact that the cavity field cannot change on the timescale of a single round trip, but as a result the LLE necessarily neglects all dynamics that might have some periodicity at the round-trip time; the fundamental timescale of LLE dynamics is the photon lifetime. 

\section{Outputs of nonlinear optics in microresonator frequency combs}

The LLE 

%\section{Frequency comb nonlinear dynamics in microresonators}
%
%The first report of frequency-comb generation in silica microcavities by Del'Haye et al initiated a serious research effort to understand how to make these combs useful for applications. The early years of this effort brought reports of comb generation in resonators made from amorphous \cite{stuff} and crystalline \cite{stuff} materials, with 



%
%
%
%
%
%
%that exhibit the  nonlinearity, which can be the dominant source of nonlinearity in centro-symmetric materials in which the second-order nonlinearity $\chi{(2)}$ must vanish. The $\chi^{(3)}$ nonlinearity, or Kerr nonlinearity, arises from the term in the Taylor expansion for the polarization response of the medium that scales with the optical intensity: $P_{NL}=...+\epsilon_0 \chi^{(3)} E^3+...$. \cite{something}. This can equivalently and usefully be described as a dependence of the refractive index on the intensity via .
%
%The third-order 
%
%The resonator quality factor is an important figure of merit for the use of optical resonators as platforms for nonlinear optics. This is discussed extensively below; typically the threshold power for nonlinear optics scales as $Q^{-2}$, meaning that in the design of practical platforms targeting high $Q$ is an important consideration. Fabrication of ultrahigh-Q resonators has been achieved with a variety of designs and materials, including ... 
%
%
% These resonators can be constructed with extremely high quality factors, upwards of $10^8$, which facilitates high circulating intensity. 
% 
% Combs using optical ring resonators leverage this confinement and the resulting long photon lifetimes in high-quality factor resonators 