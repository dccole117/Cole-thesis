 \chapter{Introduction}
\label{ch:intro}

The invention of the optical frequency comb two decades ago provided a revolution in precision measurement by dramatically improving the resolution with which we can conveniently measure time \cite{Diddams2000,Jones2000,Udem2002,Hall2006,Hansch2006}. This revolution came about through the development of a simple scheme (that required markedly \textit{less} simple advancements in capabilities in nonlinear optics\cite{Ranka2000}) by which the hundreds-of-terahertz-scale optical frequencies of a mode-locked laser could be effectively measured by electronics with bandwidth limitations on the gigahertz scale. Optical frequency combs have played an integral part in experiments and applications in contexts ranging from record-setting optical clocks, systems for ultra-low-noise microwave synthesis, broadband spectroscopy applications, and stable long-term calibration of astronomical spectrographs for exoplanet detection\cite{Steinmetz2008}. Further development of the technology beyond the first stabilization of the Ti:sapphire laser that heralded the frequency comb's arrival has enabled frequency combs to reach applications across many wavelength bands. The technology is reaching maturity, and frequency combs have been commercially available for some time.

In the last decade, methods for generating optical frequency combs without a mode-locked laser have suggested the possibility of bringing their capabilities to a wide set of applications outside the controlled environment of the research laboratory. These new frequency combs come with higher repetition rates, which makes them particularly attractive for applications where high power per comb mode, individual accessibility of comb modes, and fast acquisition times are desired; these applications include arbitrary microwave and optical waveform generation, telecommunications, and broadband, fast-acquisition-time spectroscopy. Moreover, these combs come with lower size, weight, and power (SWAP) requirements, which will enable them to bring the features that make mode-locked laser-based combs attractive into the field, enabling e.g. direct optical frequency synthesis on a chip \cite{Spencer2018}.

This thesis focuses on this second generation of optical frequency combs. The bulk of the thesis covers microresonator-based frequency combs, and especially the nonlinear dynamics involved in the generation of these frequency combs via the Kerr nonlinearity. The penultimate chapter presents a second method for generating a high-repetition-rate frequency comb without modelocking that is based on active modulation of a seed CW laser and subsequent nonlinear spectral broadening. In the final chapter, I present experimental and theoretical investigations of a technique for repetition-rate reduction of frequency combs, which may prove useful for adapting low-SWAP combs and their intrinsically high repetition rates to some applications as the technology continues to develop.

In the remainder of this chapter, I discuss the basic properties of frequency combs and explain how the optical frequencies making up a comb can be fully determined by electronics operating with gigahertz-scale bandwidths.

\section{Optical frequency combs}

An optical frequency comb is obtained by fully stabilizing the spectrum of an optical pulse train. The first frequency combs came about through full frequency-stabilization of modelocked lasers; this thesis focuses on frequency combs with pulse trains generated through other means.

\subsection{Optical pulse trains and their spectra}

In the time domain, a frequency comb consists of a train of uniformly spaced optical pulses arriving at the pulse train's repetition rate $f_r$. These pulses are typically very short compared to their repetition period $T=1/f_r$. In the frequency domain, the comb consists of a set of modes that are spaced by $f_r$ in frequency and that have amplitudes determined by an overall spectral envelope centered at the optical carrier frequency, with bandwidth inversely related to the temporal duration of the pulses. The usual description of a frequency comb, which is natural for modelocked-laser-based combs that are not derived from a CW laser, gives the frequencies of these modes as 
\begin{equation}
\nu_n=nf_r+f_0, \label{eq:combfreqsold}
\end{equation} 
where $n\sim f_{carrier}/f_r$ for the optical modes that make up the comb and $f_0$ is the carrier-envelope offset frequency, which may be defined to be between $0$ and $f_r$. The offset frequency results from the pulse-to-pulse evolution of the carrier wave underneath the temporal intensity envelope of the pulses due to a difference in group and phase velocities. An equivalent representation of the frequencies of the comb that is more natural for frequency combs directly derived from a CW laser, as described in this thesis, is
\begin{equation}
\nu_\mu=\nu_c+\mu f_r, \label{eq:combfreqsnew}
\end{equation} 
where $\nu_c$ is the frequency of the CW laser, the `pump' or `seed' laser, from which the frequency comb is derived and $\mu$ is a pump-referenced mode number, in contrast with the zero-referenced mode number $n$ of Eq. \ref{eq:combfreqsold}. Fig. \ref{fig:CombBasics} depicts the properties of a frequency comb in the time domain and the frequency domain.

It is useful to consider a mathematical treatment of an optical pulse train to understand the relationships presented above. In the time domain, the electric field $E(t)$ of the pulse train consists of optical pulses that arrive periodically and have baseband (centered at zero frequency) field envelope $A(t)$ multiplying the carrier wave of angular frequency $\omega_c$:
\begin{equation}
E(t)=\sum_{k=-\infty}^{\infty} A(t-kT)e^{i\omega_c t}. \label{eq:pulsetrain}
\end{equation}
Here, $T$ is the repetition period of the pulse train. Eq. \ref{eq:pulsetrain} can be viewed as describing a laser of angular frequency $\omega_c$ with a time-varying amplitude. This temporal modulation leads to the distribution of the power across a spectrum whose width scales inversely with the temporal duration of $A$. Intuitively, the spectrum of the comb is the spectrum of the periodic baseband field envelope $\Sigma_k A(t-kT)$, shifted by the multiplication with $e^{i\omega_c t}$ so that it is centered around the optical carrier. More formally, we can calculate the spectrum $|\mathcal{F}\left\{E\right\}|^2$ by calculating
\begin{equation}
\mathcal{F}\left\{E\right\}(\omega)\sim\left(\sum_{k=-\infty}^{\infty}\mathcal{F}\left\{A(t-kT)\right\}\right)*\delta(\omega-\omega_c),
\end{equation}
which results from the convolution (denoted by $*$) theorem for Fourier transforms. We use the Fourier transform's property that a temporal translation results in a linear spectral phase shift to obtain:
\begin{equation}
\mathcal{F}\left\{E\right\}\sim\left(\mathcal{F}\left\{A\right\}\times\sum_{k=-\infty}^{\infty}e^{-i\omega kT}\right)*\delta(\omega-\omega_c).
\end{equation}
The quantity $\Sigma_ke^{-i\omega kT}$ is the Fourier-series representation of the series of $\delta$-functions \mbox{$\Sigma_\mu\delta(\omega-2\pi\mu/T)$}, so we get
\begin{equation}
\mathcal{F}\left\{E\right\}(\omega)\sim\left(\mathcal{F}\left\{A\right\}\times\sum_{\mu=-\infty}^{\infty}\delta\left(\omega-2\pi \mu/T\right)\right)*\delta(\omega-\omega_c),
\end{equation}
and performing the convolution leads to the replacement of $\omega$ with $\omega-\omega_c$, leading to:
\begin{equation}
\mathcal{F}\left\{E\right\}\sim\sum_{\mu=-\infty}^{\infty}\delta\left(\omega-\omega_c-\mu\omega_r\right)\mathcal{F}\left\{A\right\}(\omega-\omega_c). \label{eq:combspectrum}
\end{equation}
This expression indicates that the spectrum of the comb has frequency content at modes $\nu_\mu=\nu_c+\mu f_r$, and that their amplitudes are determined by the spectrum of the baseband field envelope, shifted up to the optical carrier frequency $\nu_c$. This is the natural formulation in the case of a comb derived from a CW laser, but it obscures the carrier-envelope offset frequency in the difference between $\nu_c$ and the nearest multiple of the repetition rate, so that $f_0$ is the remainder of $\nu_c\div f_r$. In practice, if $f_r$ is known, then a measurement of $f_0$ is equivalent to a measurement of the frequency of the input CW laser.


\subsection{Frequency stabilization of optical pulse trains}

The scientific need for a method to measure optical frequencies motivated the development of optical frequency combs. While the measurement bandwidth of electronic frequency counters has improved since 1999, it remains limited to frequencies roughly one \textit{million} times lower than the frequency of, e.g., visible red light. Frequency combs present a method for measurement of the unknown frequency $f_{opt}$ of an optical signal through heterodyne with a frequency comb---if $f_{opt}$ falls within the bandwidth of the frequency comb, then the frequency of the heterodyne between the comb and the signal is guaranteed to be less than $f_r/2$, which is typically a frequency that can be measured electronically, at least for modelocked-laser-based combs. Therefore, if the frequencies of the comb are known, measurement of the heterodyne of the comb with the signal reveals the frequency of the signal, provided that the comb mode number $n$, as defined by Eq. \ref{eq:combfreqsold}, can be determined. This can be done via a wavelength measurement if sufficient precision is available, or by measuring the change $\partial f_b/\partial f_r=\pm n$, where $f_b$ is the measured frequency of the beat.

The unique utility of the optical frequency comb lies in the fact that measurement of the two frequencies $f_r$ and $f_0$, along with a measurement of the spectral envelope, is sufficient to determine the optical frequencies of all of the modes of the comb, thereby enabling frequency measurement of optical signals. Measurement of the repetition rates of optical pulse trains was possible for many years before the realization of optical frequency comb technology, as this can be done by simply impinging the pulse train on a photodetector. It was the confluence of several technological developments around the turn of the twenty-first century that allowed detection and measurement of the carrier-envelope offset frequency \todo{cite}, thereby enabling creation of fully-stabilized modelocked-laser pulse trains: optical frequency combs.

The carrier-envelope offset frequency of a pulse train is challenging to measure because it describes evolution of the optical carrier wave underneath the intensity envelope, and therefore cannot be measured through straightforward detection of the intensity of the pulse train. Presently, the most straightforward way to measure $f_0$ is $f-2f$ \textit{self-referencing}, which is illustrated in Fig.\ref{fig:f2fselfreferencing}\todo{make fig}. This can be performed only with a pulse train whose spectrum spans an octave---a factor of two in frequency. Given such an octave-spanning supercontinuum spectrum, a group of modes near mode number $N$ is frequency-doubled in a medium with the $\chi^{(2)}$nonlinearity\cite{Boyd2003}. This frequency-doubled light is heterodyned with the native light in the supercontinuum with mode number near $2N$. The frequency of the resulting beat $f_b$ is:
\begin{align}
f_b&=f_{doubled}-f_{native}\\
&=2(Nf_r+f_0)-(2Nf_r+f_0)\\
&=f_0.
\end{align}

Generating the necessary octave-spanning supercontinuum spectrum typically requires nonlinear spectral broadening of the pulse train after its initial generation, except for in specific, carefully engineered cases\todo{cite Tara, others}. Achieving the required degree of spectral broadening while preserving the coherence properties of the pulse train is a significant challenge---typically this requires launching a train of high energy ($\sim$1 nJ), temporally short ($\leq$ 100 fs) pulses into the spectral-broadening stage, and meeting these requirements is one of the important engineering considerations in designing optical frequency comb systems, as discussed in Chapters \ref{chap:EOMCombs} and \ref{chap:PulsePicking}. 

%\section{Emerging applications for frequency combs}
%\todo{Here i suppose I'd like to present some example applications}
%The work presented in this thesis is motivated by a set of applications that can leverage high frequency-comb repetition rate; or low frequency-comb size, weight, and power; or both. In general these applications exist outside the laboratory, in fields such as 

%------------------------------------------------------
%
%modes at the set of frequencies 
%
%
%
%Equivalently, if the function $A(t)$ is localized to a small interval around zero relative to the period $T$, this equation be written to emphasize the phase-shift between the carrier wave and the intensity envelope:
%\begin{equation}
%E(t)=\sum_{n=-\infty}^{\infty} A(t-nT)e^{i\omega_c (t-nT)}e^{in\phi_{CE}}, 
%\end{equation}
%where here the field $A(t)e^{i\omega_c t}$ is repeated every period, and is multiplied by a phase increasing incrementally by $\phi_{CE}=\omega_c T$. Eq. 
%
%
%
%\color{red}
%The spectrum of the frequency comb consists of a set of uniformly spaced optical modes at frequencies , multiplied by an overall spectral envelope centered at the optical carrier frequency and corresponding to the temporal intensity envelope of the pulses. The optical frequencies $\nu_n$ are spaced by $f_r$. The carrier-envelope offset frequency $f_0$ represents the offset of the zero\textsuperscript{th} comb mode from zero frequency, and therefore the offset of each mode $\nu_n$ from the closest harmonic of the repetition rate. This offset arises from the pulse-to-pulse evolution of the carrier wave under the pulse train's intensity envelope. 
%
%
%A significant challenge in using $f-2f$ self-referencing to measure the offset frequency of a pulse train is to generate the required octave-spanning supercontinuum spectrum with noise properties that permit detection of the beat between the $f_{doubled}$ and $f_{native}$ signals. In general, this requires nonlinear spectral broadening of the pulse train after its initial generation\todo{except Tara, and others??}, and 
%
%
%In the above, we have used the linearity and convolution properties of the Fourier transform, with convolution denoted by $*$. We have also used the Fourier transform for the Dirac comb . Eq. \ref{eq:combspectrum} directly reveals the connection between the spectrum  of the electric field of a pulse train and the baseband pulse envelope $A(t)$, the carrier frequency $f_c=\omega_c/2\pi$, and the repetition rate $f_r=1/T=\omega_r/2\pi$.
%
%The first optical frequency combs came about through full frequency-stabilization of optical pulse trains generated in modelocked lasers. A laser cavity with broadband gain can support many oscillating frequency modes; this number is on the order of ten thousand for a typical telecommunications-band fiber laser, and can be hundreds of thousands for a Ti:sapphire laser cavity. Without a mechanism to enforce a fixed relationship between the modes, the modes oscillate independently and the laser output is uncontrolled. A modelocked laser is obtained through the introduction of a modelocking mechanism that provides for lower cavity losses in pulsed operation, in which the modes oscillate together and periodically constructively interfere, relative to unsynchronized multi-mode operation. Common modelocking mechanisms are saturable-absorber mirrors, Kerr-lens modelocking, and nonlinear polarization-rotation modelocking.
%
%The pulse train generated in a modelocked laser can be described equally well in either the time domain or the frequency domain. 
%
%
%The spectrum of the pulse train consists of a set of equidistant modes with optical frequencies described by $\nu_n=nf_r+f_0$. Here $\nu_n$ is the frequency of the $n$\textsuperscript{th} mode, referenced to zero frequency; $f_r$ is the pulse train's repetition rate, and is the separation between adjacent modes in the frequency domain, and $f_0$ is the 'carrier-envelope-offset frequency,' denoting 
%
%
%modelocked laser consists of many oscillating modes, supported by broadband laser gain, that have a fixed phase relationship imposed by a modelocking mechanism. In a typical 250 MHz repetition-rate optical pulse train in erbium-doped fiber, this number is on the order of ten thousand.  These pulse trains result from synchronization of many oscillating laser modes in a cavity with broadband gain through the introduction of a modelocking mechanism. A modelocked laser consists of a laser cavity having 1. broadband gain and 2. a modelocking mechanism. generate an optical pulse train when it has broadband gain in the presence of a modelocking mechanism. If a laser is made to oscillate under these conditions without a modelocking mechanism, these modes can 'lase' independently. This generates an output waveform that appears random, resulting from the superposition of these thousands of modes and their individual frequency fluctuations. If, on the other hand, modelocking is enforced so that the modes oscillate in a coherent fashion, their electric field frequency components periodically constructively interfere, yielding a train of pulses. This is depicted schematically in Fig. \cite{fig:MLnoMLpulsetrains}. To induce modelocking, a mechanism that favors higher peak-intensity pulsed operation is introduced into the cavity. Two common modelocking mechanisms are Kerr-lens based modelocking, in which the spatial Kerr effect focuses a  beam through a small aperture in the laser cavity more effectively at higher power, and semiconductor saturable absorber mirrors that have higher reflectivity at higher incident intensities. Modelocking can be used to generate laser pulses that are on the order of hundreds of femtoseconds long, or even shorter.
%\color{black}





%This first-generation frequency comb technology is reaching maturity, and mode-locked laser-based frequency combs can now be purchased commercially from several providers. 
%
%\color{red} Need to motivate microcombs, EOMC, pulse picking. HRR combs in general from standpoint of new applications, also comes with low SWAP (but perhaps this is weak because for example spiral resonators can make low RR combs in small packages).
%
%HRR combs naturally give octave span, don't require external broadening 
%
%Applications for HRR combs: telecom, optical arbitrary waveform generation, spectroscopy, astrocombs \cite{Kippenberg2011}.
%
%Applications for compact combs: ranging \cite{Suh2018,Trocha2018}, field spectroscopy measurements
%\color{black}






