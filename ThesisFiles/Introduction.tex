 \chapter{Introduction}
\label{ch:intro}

\section{Motivation}
\label{sec:introMotivation}
\subsection{Seeing the invisible}
\lipsum[3] \cite{Turing1952} \lipsum[1-3]

% % % % % % Bullet through Apple
%\begin{figure}[htpb]
%	\begin{center}
%		\includegraphics[width=110mm]{\FigPath/Figures/Introduction/Edgerton.jpg}
%	\end{center}
%	\caption[Edgerton's ``Bullet Through Apple'']{\textbf{Edgerton's iconic 1964 photograph ``Bullet Through Apple'' \cite{Edgerton1964}.} Using a carefully timed xenon flashlamp in a completely dark room, Harold Edgerton was able to photograph a rifle bullet passing through and apple with microsecond time resolution. \copyright2010 MIT. Courtesy of MIT Museum }
%	\label{fig:introEdgerton}
%\end{figure} 


%\marginpar{\hangindent=0.2cm \underline{Speed of light}\\300 Mm/s\\300 km/ms\\300 m/$\mu$s\\300 mm/ns\\300 $\mu$m/ps\\\textbf{300 nm/fs}\\300 pm/as}

\subsection{Femtosecond lasers}
\lipsum[66]





% % % % % % Quantum dots
%\begin{figure}[htpb]
%	\centering
%	\begin{subfigure}[b]{0.5\textwidth}
%		\includegraphics[width=\textwidth]{\FigPath/Figures/Introduction/qds.jpg}
%	\end{subfigure}%
%	~
%	\begin{subfigure}[b]{0.5\textwidth}
%		\includegraphics[width=\textwidth]{\FigPath/Figures/Introduction/QD3d.png}
%	\end{subfigure}%
%	\caption[Various sizes of CdSe quantum dots]
%	{\textbf{Cadmium selenide quantum dots (CdSe QDs).} (left) Each vial contains the same chemical composition (CdSe QDs and toluene), but the different sizes cause the nanocrystals to fluoresce different colors. The bandgap increases with decreasing particle size, so the vials on the left contain the smallest QDs (largest bandgap) and the vials on the right contain the largest QDs (smallest bandgap). Image courtesy of NN-Labs Inc. (right) An illustration of CdSe quantum dots shows the individual atoms clustered into faceted nanocrystals.}
%		\label{fig:introQDs}
%\end{figure}

% % % % % % Three-step model
%\begin{figure}[htpb]
%	\begin{center}
%		\includegraphics[width=150mm]{\FigPath/Figures/Introduction/three-step.pdf}
%	\end{center}
%	\caption[The three-step model of high harmonic generation (HHG)]
%	{\textbf{The semi-classical three-step model of high harmonic generation (HHG).} The blue line shows the combined energy of the laser and Coulomb potential, which is changing as a function of time. The green line shows the total energy of the electron (kinetic energy plus potential energy of the laser field) plotted as a function of the distance from the ion. (a) In the presence of a strong laser field (in this case $1\times10^{14} \wcmsq$), the Coulomb potential of an atom (or molecule) is strongly deformed. Under these conditions, a valence electron can tunnel through the barrier and emerge into the continuum (Step 1). (b) In Step 2, the free electron is now driven by the laser field, first away from the ion and then (depending on the phase of the field when the electron tunnels) driven back towards the ion. (c) In Step 3, the electron recombines with the ion emitting a high-energy photon. In the case of strong-field ionization (SFI), the electron does not recombine, but can scatter from the ion.}
%	\label{fig:introThreestep}
%\end{figure} 

















