 \chapter{PM Pumping} \label{ch:PMPumping}




We demonstrate protected single-soliton formation and operation in a Kerr microresonator using a phase-modulated pump laser. Phase modulation gives rise to phase and amplitude variations in the resonator background, which in turn lead to an operation regime in which multi-soliton degeneracy is lifted and a single soliton is the only observable behavior. Direct excitation of single solitons is indicated by observed reversal of the characteristic ‘soliton step.’ Phase modulation also enables precise control of the soliton pulse train’s properties, and measured dynamics agree closely with simulations. We show that the technique can be extended to high repetition-frequency Kerr solitons through subharmonic phase modulation. These results facilitate straightforward generation and control of Kerr-soliton microcombs in integrated photonics systems. 



Dissipative temporal cavity solitons in Kerr microresonators [1–3] have the potential to provide the revolutionary capabilities of frequency combs in a chip-integrable platform. This would extend the reach of frequency combs to applications in communications,  computation, and sensing with low size, weight, and power. Progress has come rapidly in the field of microresonator-soliton-based frequency combs, but for these combs to reach applications, simple, repeatable, and platform-independent methods of soliton generation and control are needed. The basic challenge is that solitons in microresonators are independent excitations, and a resonator can host zero, one, or many co-circulating solitons at a given pump-laser power and frequency, with each soliton giving rise to its own out-coupled pulse train. Further, under normal conditions solitons can only be generated by condensation from extended modulation-instability (MI) patterns (primary comb/Turing patterns, or noisy comb/spatiotemporal chaos) that provide appropriate initial conditions. Thermal stability must be maintained during the drop in intracavity power associated with the transition from a high duty-cycle MI pattern to a low duty-cycle soliton. A variety of schemes have been demonstrated to address these challenges and obtain single solitons [4–7], and many achieve excellent performance. In general these schemes increase experimental complexity, exploiting non-adiabatic variations in pump-laser power and frequency, and involve at least some amount of stochastic fluctuation in the output. 
One notable possibility is modulation of the pump laser at a frequency near the resonator free-spectral range (FSR) [8–11], which can enable deterministic condensation of either one or zero solitons from an MI pattern. Further, it has been demonstrated that phase modulation (PM) can facilitate generation and control of single solitons [10,12,13]. Here we use PM at the FSR to deterministically excite single solitons directly from a chirped background that is stable elsewhere, as proposed in Ref.  [14]. Exiting the resonator is a train of solitons spaced by the round-trip time, as shown in Fig. 1a.  Importantly, no transient perturbation to the system parameters is required.
Our results demonstrate a regime in which single-soliton operation is fundamentally protected, without the degeneracy between N=0,1, and many solitons that exists for a continuous-wave (CW) pump laser.  To motivate the experimental work that follows, we present theoretical results that illustrate the utility of a PM pump. We use the nonlinear partial-differential Lugiato-Lefever equation (LLE) with modification of the driving term for phase-modulation of depth δ_PM [1,14–17]:


The normalized quantities used in the LLE are defined as follows [15]: ψ is the envelope for the intracavity field normalized so that |ψ|^2=1 at the absolute threshold for parametric oscillation; τ=t/2τ_γ, where t is the time and τ_γ=1/2πΔν is the cavity photon lifetime and Δν is the cavity resonance linewidth; α=2(ν_0-ν_pump )/Δν is the detuning between the pumped resonance with frequency ν_0 and the pump laser with frequency ν_pump; F is the pump strength normalized so that F^2=1 at the absolute threshold for parametric oscillation; and β_2=-2D_2/Δν<0 is the anomalous resonator dispersion, with D_2/2π=∂^2 ν_μ/∂μ^2 |_(μ=0)>0, where ν_μ represents the set of cavity resonance frequencies. The azimuthal angle θ co-rotates at the frequency f_PM, which is presently assumed to be equal to f_FSR.


We perform simulations of the LLE to investigate soliton degeneracy for the range of pump-laser detunings overwhich solitons exist. These simulations use a fourth-order Runge-Kutta algorithm in the interaction picture [18] with adaptive step size [19]. A comparison of the resulting soliton energy-level diagrams for the CW case (δ_PM=0) and the PM case (δ_PM=π/2) are shown in Fig. 1b. We find that PM transforms the resonator excitation spectrum from a series of N=0,1,2,…,N_max solitons to a single level N=1 near threshold, eliminating degeneracy between these states. This occurs due to amplitude variations resulting from the phase modulation, with dispersion and nonlinearity providing PM-to-AM conversion. We can gain some insight into the origin of this effect by inserting the ansatz ψ(θ,τ)=ϕ(θ,τ)e^(iδ_PM  cos⁡θ ) into Eq. (1) [13].  By expanding the second-derivative term and setting derivatives of ϕ to zero we arrive at an equation for the quasi-CW background in the PM-pumped resonator:
F=(γ(θ)+iα_eff (θ))ϕ-i|ϕ|^2 ϕ,	(2)
where we have defined effective loss and detuning terms:
γ(θ)=1+β_2/2 δ_PM  cos⁡θ,	(3)
α_eff (θ)=α-β_2/2 δ_PM^2  sin^2⁡θ.	(4)





This equation yields an approximate solution for ψ:
ψ=(Fe^(iδ_PM  cos⁡θ ))/(γ(θ)+i(α_eff (θ)-ρ(θ)) ),	(5)
where ρ(θ)=|ϕ(θ)|^2 is the (smallest real) solution to the cubic polynomial in ρ that results from calculating the modulus-square of Eq. (2). In neglecting the spatial derivatives of ϕ but retaining the derivatives of the phase term e^(iδ_PM  cos⁡θ ) we have made the approximation that the dominant effect of dispersion comes from its action on the existing broadband phase-modulation spectrum. This model reveals that amplitude variations in the quasi-CW background can be expected as a result of the spatially-varying effective loss and detuning terms that arise from the periodically-chirped pump laser.
Fig. 1c compares the predictions of numerical LLE simulations (color) with the analytical model (black). The two agree quantitatively at small modulation depth (δ_PM=π/2, blue) and qualitatively at larger depth (δ_PM=4π, green). Both the simulations and the approximate analytic solution indicate that the background has two peaks per round trip in the presence of phase modulation, which suggests a mechanism for spontaneous single-soliton generation: At threshold the larger peak becomes locally unstable, and a soliton forms [20,21]. Moreover, it is known that if solitons exist elsewhere they are pushed to the larger peak by the background’s modulated phase [13]. This makes superpositions of N>1 solitons unstable and practically forbidden. Generation of single solitons then simply requires tuning the pump power and frequency to appropriate values, regardless of initial conditions. 
The detuning for soliton generation can be estimated using Eq. (2) by calculating the value of α where ρ(θ=0)=1. This comes with a further approximation—simulations reveal that the critical detuning for soliton formation is near but not necessarily at ρ=1 because the spatial interval over which threshold is exceeded must have some minimum width. However, this approach quantitatively captures the behavior shown in Fig. 1b, predicting soliton generation at α=2.737.
Because it is background intensity variations that result local MI and spontaneous soliton generation, it is natural to consider whether a similar technique can be employed using amplitude modulation (AM). In principle this can be done. However, to experimentally realize the robust single-soliton operation we describe below, the intensity modulation would need to yield a narrow cavity intensity maximum, and would therefore necessarily be broadband (e.g. pulsed pumping [11,21]), or would require supplementary PM to generate a trapping site for solitons. We are unaware of a straightforward implementation of AM for protected single-soliton operation that rivals the simplicity of the scheme presented here.
We implement the approach described above to realize completely deterministic generation of single solitons without condensation from an extended pattern, summarized in Fig. 2. We use a 22 GHz-FSR silica ring resonator with Δν~1.5 MHz linewidth [22], pumped by a laser with normalized power F^2 between 2 and 6 that is phase-modulated at a rate f_PM~22 GHz with relatively small depth δ_PM~π. We overcome thermal instabilities [23] and control the detuning ν_0-ν_pump in real time using a frequency-agile pump [24], with an AOM-shifted probe for continuously monitoring the detuning. We decrease the detuning from a large initial value (~40 MHz), and a soliton is generated near 5 MHz (dependent upon the pump power and coupling condition). Measuring the power converted through FWM to new frequencies, the ‘comb power,’ reveals a step upon soliton formation, shown in Fig. 2a. This represents a reversal of the characteristic ‘soliton step’ that typically signals condensation of solitons from an extended pattern and indicates direct generation of a soliton from the background. After soliton generation, α may be increased again without loss of the soliton, consistent with Fig. 1b. We have verified that it is possible to turn off the PM while preserving the soliton (see also Ref.  [14]). 



Automating soliton generation by repeatedly scanning the laser into resonance (detuning ~5 MHz) and back out again (20 MHz, far enough that the soliton is lost) has enabled reversible generation of 1000 solitons in 1000 trials over 100 seconds, with a 100 % measured success rate. Our probe-laser setup enables measurement of the detuning at which soliton generation occurs, which changes little from run to run. Fig. 2c presents a histogram of measurements for the generation of 160 solitons.
Besides enabling protected sigle-soliton operation, PM pumping  also naturally provides timing and repetition-rate control, because the solitons are pushed towards the intracavity phase maximum [13]. This is illustrated in Fig. 3. In our experiments, the repetition rate of the out- coupled pulse train (f_rep) remains locked to f_PM over a bandwidth of ~±40 kHz. In Fig. 3a, we show a measured spectrogram of f_rep as f_PM  is swept sinusoidally over ±50 kHz. The repetition rate follows the PM except for glitches near the peaks of the sweep. In the inset of Fig. 3a we overlay the results of LLE simulations (see below) that qualitatively match the observed behavior. These simulations indicate that the periodic nature of the glitches is due to the residual pulling of the phase modulation on the soliton when the latter periodically cycles through the pump’s phase maximum. Our observed locking range of ~±40 kHz agrees well with an estimate δ_PM×D_2/2π~44 kHz [13] using the approximate measured value D_2=14 kHz per mode.
Measured eye diagrams presented in Figs. 3b and 3c illustrate the switching of f_rep as f_PM is switched by 80 kHz around the soliton’s natural repetition rate. In Fig. 3b, f_PM is switched with 200 μs period and 10 μs transition time; in Fig. 3c it is switched with 100 μs period and 60 ns transition time. This data is obtained by detecting f_rep and passing the signal through two paths, one with an element that induces a frequency-dependent phase shift. From the resulting phase difference the repetition rate can be measured in real time. These eye diagrams show that the PM enables exquisite control of the soliton pulse train.
We further explore the dynamics of repetition-rate switching by performing additional LLE simulations. We introduce the term +β_1  ∂ψ/∂θ to the right-hand side of Eq. (1), where β_1=-2(FSR-f_PM)/Δν represents a difference between the modulation frequency and the FSR 



of the resonator near the pump wavelength [13,14];  β_1 may be varied in time.  In Fig. 3c we overlay a simulation of switching conducted for parameters (Δν=1.5 MHz, δ_PM=0.9 π) near the experimental values, and the agreement between measurements and simulation indicates that the measurements are consistent with fundamental LLE dynamics. We present the results of additional simulations in Fig. 3d; the basic observation is that the switching speed of f_rep is limited by the resonator linewidth, and can be modestly improved by increasing δ_PM.
One apparent barrier to the use of a phase-modulated pump laser for protected single-soliton generation and manipulation is the electronically-inaccessible FSRs of some typical microcomb resonators. However, it is possible to overcome this challenge by phase modulating at a subharmonic of the FSR.  Simulations indicate that PM can directly excite single solitons with small modulation depth, e.g. δ_PM=0.15π. In this limit, only the first-order PM sidebands are relevant, and their amplitude and phase relative to the carrier control the dynamics. For a small desired modulation depth defined by the relationship between the first-order sidebands and the carrier, it is possible to modulate


at a frequency ~f_FSR/N so that the N^th-order PM sidebands and the carrier address resonator modes with relative mode numbers -1, 0, and 1. The depth of modulation at the frequency f_FSR/N can be chosen to fix the amplitudes of the N^th-order PM sidebands relative to the carrier and target a desired effective modulation depth. It is worth noting that when N is odd, phase modulation is recovered when the sidebands of order -N,0, and N address resonator modes -1, 0, and 1. When N is even the result is pure amplitude modulation, such that the driving term takes the form F(1+A cos⁡θ). Under some circumstances this AM profile also enables spontaneous single-soliton generation, but we note that this modulation profile cannot be obtained from a standard Mach-Zehnder modulator, which provides a drive like F cos⁡(η+δ cos⁡θ ).
Fig. 4 presents an example of this technique. We simulate protected single-soliton generation with PM at f_mod=f_rep/N=f_rep/21. The effective modulation depth is 0.15π, which requires real modulation depth at the frequency f_mod with depth δ_PM~8.3π.  Because the phase modulation spreads the optical power into the PM sidebands, use of this technique requires higher optical power for the same effective pumping strength; in this example the optical power must be increased by ~15.6 dB. While the required modulation depth and pump power are higher with subharmonic phase modulation, neither is impractical. This technique could be used for protected single-soliton generation in high-repetition rate systems; the example above indicates that it could be immediately applied to deterministic single-soliton generation in a 630 GHz-FSR resonator with 30 GHz phase modulation. 
In this work, we have shown that PM-pumping fundamentally changes a resonator’s excitation spectrum and enables a new regime of protected single-soliton operation. The technique is applicable to resonators with electronically accessible f_rep, which are important components of proposals for photonic integration of Kerr-solitons  [25,26], and can reach higher repetition-rate systems via subharmonic modulation. After soliton generation, the PM can optionally be turned off, recovering the properties of the non-PM soliton. We expect this technique to enable new experiments. For example, PM-pumped solitons are generated with known absolute timing, enabling immediate transduction of the modulation phase onto




the pulse train; this is impossible with solitons stochastically condensed from an extended pattern. Our work brings microresonator solitons closer to applications.
