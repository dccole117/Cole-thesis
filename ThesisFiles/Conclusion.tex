 \chapter{Summary and outlook}\label{chap:conclusion}

This thesis has described advances in systems for generation of optical frequency combs derived from a continuous wave laser. A major focus of the thesis has been microresonator-based frequency combs. I also described generation of a frequency comb by active-modulation of a CW laser, and I presented results on the downsampling of frequency combs for repetition-rate reduction.

In the context of microresonator-based frequency combs, I described three results: 1. The investigation and implementation of a technique for spontaneous soliton generation in Kerr resonators using a phase-modulated pump laser, 2. The observation and explanation of soliton crystals in Kerr resonators, and 3. A theoretical investigation of Kerr-comb generation in Fabry-Perot cavities, with an emphasis on the properties of solitons and soliton generation. These results all help to more clearly define what is possible with these systems, and suggest avenues for further research. 

Soliton generation with a phase-modulated pump laser is a promising candidate for the mechanism by which Kerr-soliton combs can be generated deterministically on chip. Two directions for continued work are additional theoretical investigations of the full LLE with a phase-modulated pump, which could provide insight into the dynamics beyond what is possible using the approximations described in Chapter \ref{chap:PMpumping}; and implementation of the technique with resonators that have electronically-inaccessible free-spectral ranges, using the subharmonic-modulation approach that we proposed. Incorporation of the technique into a chip-integrated Kerr-soliton comb may also require modification of the technique that we demonstrated to overcome thermal instabilities associated with the increasing-frequency pump-laser scan.

The investigation of soliton crystals presented here serves several important purposes. First, it represented an important step towards full explanation of observed Kerr-comb phenomena in terms of the LLE model. \todo{am I citing pascal in sc chapter?} Second, soliton crystals have the attractive properties of single-soliton Kerr combs, with the additional property that a soliton crystal of $N$ pulses has conversion efficiency of pump-laser power into the comb that is roughly $N$ times higher than a comparable single-soliton comb. With careful preparation of a particular crystal state, this could make them attractive for applications like optical arbitrary waveform generation and nonlinear spectroscopy. Additionally, soliton crystals present a hugely degenerate configuration space that could be useful in implementations, for example, of an on-chip optical buffer or in communications applications \cite{Leo2010}. Finally, experimental generation of soliton crystals is significantly simpler than generation of single solitons, where the change in the duty cycle of the optical waveform from extended pattern to single soliton leads to thermal instabilities that are alleviated only with precise control of the pump-laser power and frequency. Thus, it is possible to propose a scheme for deterministic on-chip soliton crystal generation that makes use of two resonators, each constructed of looped single-mode optical waveguides. One resonator is pumped by a laser and hosts the soliton crystal. The second resonator need not be pumped, and exists to provide a specific perturbation to the mode structure of the first resonator to enable soliton crystallization; this could be achieved through careful engineering of the coupling between the resonators. If the free-spectral range of the second resonator is considerably higher than the free-spectral range of the first, and not near one of its harmonics, then realization of single-mode perturbation to the mode structure of the first resonator could be achieved. Implementing deterministic soliton crystal generation on a chip in this way could greatly simplify requirements on the other components in a system for full-integration of Kerr solitons, as soliton generation could be achieved through slow tuning of the pump laser.

The theoretical investigation of Kerr-comb generation in the Fabry-Perot geometry will provide useful guidance for future experimental work. An obvious direction for continued work is the generation of solitons in Fabry-Perot cavities that make use of the additional degree of freedom provided by the dispersion applied by reflection at the ends of the cavity. This would build on previous experiments \cite{Braje2009,Obrzud2017}. In fact, soliton generation in Fabry-Perot cavities constructed of potted fiber ferrules with high-reflectivity end-coatings has already been realized at NIST Boulder \cite{Zhang2018}, but there remains work to be done to achieve control the total cavity dispersion with chirped mirror-coatings. Unresolved questions include the effect of uncontrolled expansion of the mode in the coating on both the mirror reflectivity and its group-velocity disperson. Looking to the chip scale, integrated Fabry-Perot cavities constructued of single-mode waveguides with photonic-crystal mirrors is a promising route for development that would further reduce the footprint of Kerr-comb systems. This work is ongoing at NIST Boulder, and primary comb has been observed in such a cavity \cite{Yu2018}. Finally, I note that the proposal for deterministic chip-scale generation of soliton crystals presented above could be realized with two co-linear on-chip Fabry-Perot cavities, where the first cavity hosts the crystal, which is out-coupled in reflection, and the second cavity provides a perturbation to the first cavity's mode structure.







