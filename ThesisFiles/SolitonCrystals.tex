 \chapter{Soliton Crystals} \label{ch:SolitonCrystals}

This chapter presents results on the self-organization of ensembles of soliton in optical microring resonators. The reported phenomenon explains physics that goes beyond the basic LLE model of Kerr-comb formation, as is described in Sec. \ref{sec:crystallizationmechanism}. We refer to these self-organized ensembles as `soliton crystals,' which extends an analogy to condensed-matter physics that has been made in other nonlinear-optical systems, including single-pass nonlinear fiber systems \cite{23} and harmonically mode-locked fiber laser \cite{24-25}, where a mechanism for soliton crystallization that is based on two distinct timescales of the laser medium and is different from the one presented here was identified. Notably, the spatiotemporal chaos exhibited in the LLE was referred to as a `soliton gas' in early studies of nonlinear dynamics in passive fiber-loop resonators \cite{20-22}. 

Soliton crystals are soliton ensembles in which each soliton lies on a lattice site $\theta_n= 2\pi n/N$ in the co-moving frame, where $N$ is the lattice parameter that arises from the fundamental physics of the system as described below and $n$ indexes over the lattice sites. We have observed a wide variety of crystal configurations, and we present some of them in Sec. \ref{sec:crystallography}; typically a small fraction of lattice sites are occupied by solitons. Soliton crystals are characterized by stable, dense occupation of the resonator by soliton pulses, and this dense occupation comes with high circulating power relative to single solitons or few-soliton ensembles. This important fact allows soliton crystals to be generated with decreasing pump-laser frequency scans across the resonance that are adiabatic in the sense that both the resonator temperature and the intracavity waveform are maintained at the values\footnote{\color{red}something about chaos\color{black}} that would be expected from the instantaneous $\alpha$ and $F^2$ parameters throughout the scan.

We demonstrate generation of a soliton crystal in Fig. \ref{fig:crystalgeneration}; it is useful to contrast this with the behavior exhibited in Fig. \ref{fig:solitonsteps}. 

\section{Mechanism of soliton crystallization}

Fig. \ref{fig:crystal1} presents the optical spectrum and corresponding time-domain simulation of a soliton crystal. This spectrum consists of widely-spaced primary-comb lines that are separated by many resonator FSR, superposed on top of an underlying $\mathrm{sech}^2$ spectrum of the kind that corresponds to a single soliton. In fact, this spectrum can be understood through the basic superposition principle of the electric field: The primary comb spectrum with spacing $N\times f_{FSR}$ corresponds to a train of $N$ uniformly spaced pulses in the resonator. The observed crystal spectrum corresponds to such a pulse train with a single vacancy, where a pulse is missing. The effect of this vacancy on the spectrum can be understood by considering the addition of an \textit{out-of-phase} pulse to the pulse train that coincides in time with one of the existing pulses---in the time domain this corresponds to removal of one of the pulses, while in the spectral domain this corresponds to the addition (in the phase-sensitive field quantity) of the primary-comb spectrum and the single, out-of-phase soliton. 

The simulated time-domain waveform of the soliton crystal presented in Fig. \ref{fig:crystal1} is not stable---the observed width of the spectrum fixes the ratio between $\alpha$ and $\beta$, as seen from Eq. \ref{eq:LLEsoliton}. This ratio then fixes the temporal duration of the solitons, in turn determining the characteristic length of their interactions. When an attempt to simulate the crystal according to the LLE is made with parameters $\alpha=??$ and $\beta=??$ that give agreement with the measured width of the optical spectrum, it is found that pair-wise attractive interactions between solitons lead to collapse of the crystal.

We have identified a stabilization mechanism in the form of the effect of avoided mode-crossings in the resonator spectrum on the soliton waveform. In a multi-mode resonator 